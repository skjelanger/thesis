\documentclass[main.tex]{subfiles}

\begin{document}
\chapter*{Summary and Outlook}\label{cpt:sumout} \addcontentsline{toc}{section}{\nameref{cpt:sumout}}

\subsection*{Motivation}
The motivation of the thesis was to investigate how the distribution of leading partons behave inside of a QCD jet, and whether it can be described and solved analytically. Since the most energetic parton in a shower is (presumably) less sensitive to medium fluctuations, it could serve as a cleaner probe of the QGP, and therefore improve studies of jet quenching and QGP. To thoroughly acquaint ourselves with the required analytical and numerical methods of parton showers, we had to start by examining the inclusive parton distribution which is well studied. The presented methods could then be verified by comparing the numerical results and the analytical solutions with one another.

\subsubsection*{Discussion and results}
In \autoref{cpt:ana} we showed how the inclusive distribution in vacuum is described by the DGLAP evolution equations. Our treatment focused on the leading order behavior. A suitable evolution variable was introduced to simplify the evolution equations and impose angular ordering in our showers. Writing them as integral equations allowed us to interpret the Sudakov form factor as a branching probability in a given interval. Solving the DGLAP equations was possible by considering gluon showers in the small \(x\) limit. This was done by performing a Mellin-transformation and then using the saddle-point approximation to find a solution that can be transformed back into momentum space. 

When discussing jet evolutions in medium, we chose to start at the BDMPS-Z spectrum which describes the induced soft gluon radiation of jets. When incorporating these soft emissions into the evolution equations we obtain the in-medium kinetic rate equations, which is the medium counterpart to the DGLAP equations. Focusing on gluons, and using the reduced splitting kernel, the evolution equation was solved in Laplace space. The solution obtained required no further assumptions or constraints and is therefore a valid solution for how a medium cascades consisting exclusively of gluons evolve with the reduced kernel.

Implementing the vacuum and medium evolution equations into Monte-Carlo programs was done in \autoref{cpt:num}. The general procedure was the same for both evolutions. By generating expected branching intervals from the Sudakov form-factor, the evolution boundaries can be implemented, and when a parton is selected for branching we can sample a random splitting value by using the Metropolis-Hastings algorithm.

Plotting the results of the Monte-Carlo programs alongside the analytical results allowed us to discuss the properties of the cascades, and highlight differences and discrepancies. For gluon cascades in vacuum, \autoref{fig: vacuum_gluons_MCandAnalytical_comparison_lin} showed how the solution of the DGLAP equation is in good agreement with the Monte-Carlo in the small \(x\) and large \(t\) limit, as expected. The behavior of vacuum cascades with both quarks and gluons was presented in \autoref{fig: vacuum_distribution_quark_and_gluon} for showers where the initial parton was either a quark or a gluon. This was also the first indication of how the leading parton distribution behaves, clearly showing in \autoref{fig: vacuum_distribution_quark_and_gluon} that the leading and inclusive distributions are identical for values of \(x>0,5\). 

The Monte-Carlo generated distribution for gluon cascades in vacuum was plotted alongside the analytical solution of the in-medium kinetic rate equation in \autoref{fig: medium_gluoncomparison_analytical_lin}. The two graphs were in good agreement for all values of \(\tau\) and \(x\). In \autoref{fig: medium_comparison_scaling} the Monte-Carlo was run to high values of \(\tau\) to display the scaling property of the BDMPS-Z spectrum.

Finally, having built a deeper understanding of parton showers, both analytically and numerically, we moved on to the leading parton distribution in \autoref{cpt:leading}. Using a known model for the energy-loss of the leading parton we calculated a solution valid in the small \(x\) limit. The resulting plot in \autoref{fig: leading_supersimple1} confirmed that it is a good fit for values \(x>0,8\). In order to suggest a better model for the energy-loss, which should be valid for harder gluon emissions, we must know the validity range of the model. We attempted to address this issue by proposing the concept of on and off-branch leading partons, and then investigating how often the leading parton remains on-branch. The results showed that up to values of \(\tau \sim 0.5\), the majority (\(\sim 80\%\)) of the leading partons are on-branch. 

The method of generating functionals was used to formulate the evolution equations for the leading parton distribution in vacuum, valid for on-branch leading partons. The key difference from the inclusive distribution is that we must assume that the emitted parton has a momentum \(z<1/2\), such that the parton we are following is always the hardest. When only branching from the leading parton we are also keeping it on-branch, which is where the on-branch discussion becomes relevant. The evolution equations were then written in Mellin space and solved. Unfortunately returning to momentum space is challenging, and further development has not been made. 

\subsection*{Outlook}
The list of points which could be improved for our Monte-Carlo programs is long. Examples include using the full splitting functions in medium, and including both quarks and gluons in medium. But improving the Monte-Carlo is altogether not all that interesting as it has already been done for event-generators such as PYTHIA. The reason for us to develop them from the ground up was to illustrate how the solution to the evolution equations, while difficult to solve analytically, practically appears out of nowhere when applying a Monte-Carlo approach. Another approach which might be able to validate the leading parton evolution equations, is to implement them into a leading parton Monte-Carlo shower, and compare the results with the leading parton distribution of the other shower programs.

There are several results for the leading parton distribution we would have liked to obtain in \autoref{cpt:leading}. The most obvious one is that the solution obtained in Mellin space for the leading parton evolution equation is difficult to transform back to momentum space. The consequence is that the evolution equation can not be compared directly with the results of our Monte-Carlo parton showers. 

Another issue, which is a bit more subtle, is that we would ideally have wanted an evolution equation for the leading parton in medium, as this allows us to compare with the current energy-loss distribution. This challenge is closely related to a general search for a more precise formulation of the energy-loss of a parton traversing a medium. Such an expression would pave the way for improved calculations of the nuclear modification form factor. Further work could also extend the formalism to include heavy quarks. 



\end{document}