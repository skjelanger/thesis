\documentclass[main.tex]{subfiles}
\begin{document}

\addtocontents{toc}{\let\protect\contentsline\protect\nopagecontentsline}
\chapter{Analytical}\label{cpt:ana}
\addtocontents{toc}{\let\protect\contentsline\protect\oldcontentsline}
This chapter will investigate parton cascades in both vacuum and medium, from an analytical standpoint. Properties of the two cascades will be discussed and compared against one another, the evolution equations will be presented alongside the respective splitting functions, and finally we will find a solution to the evolution equations. The most interesting observable in this chapter is the inclusive parton distribution, initiated by a single quark or gluon in the aftermath of some collision. 
\section{Formalism of Parton Branching in Vacuum}
This section will cover the analytical details related to parton showers in vacuum. Beginning by defining an evolution variable that will ensure angular ordering for our showers and simplify the DGLAP evolution equations. After an overview of the DGLAP equations, they will be rewritten in terms of a Sudakov form factor, which will be a key part of creating our parton shower programs. At the end of the chapter, a solution to the DGLAP equation will be presented.

\subsection{Properties of vacuum cascades}
\subsubsection*{Evolution variable}
Before introducing the evolution equations, we will define an evolution variable in order to simplify the evolution equations, and impose angular ordering in our showers~\cite{Dasgupta_2015}. The cross section for branching from one to two partons was calculated in \autoref{eqn: branching_cross_section}, and it can be interpreted as a probability. With a fixed coupling \(\alpha_s << 1\), the probability of branching can be written as
\begin{align}
    d\mathcal{P}_{1\rightarrow 2} &= \frac{\alpha_s}{\pi} \frac{d\theta}{\theta} \, P(z)\, dz.
\end{align}
The probability of branching changes in a given volume element by
\begin{equation}\label{eqn: evolutionvariable_intermediate_condition1}
    \frac{d\mathcal{P}}{d\theta dz} = \frac{\alpha_s}{\pi} \frac{1}{\theta} P(z).
\end{equation}
Now we want to replace \(\theta\) with an evolution variable in order to simplify \autoref{eqn: evolutionvariable_intermediate_condition1}. This evolution variable will be used for all vacuum cascades from this point forward, and can be written as
\begin{equation}\label{eqn: evolution_variable_vacuum}
    t = \frac{\alpha_s}{\pi} \int_{\theta_{\text{min}}}^{\theta} \frac{d\theta'}{\theta'}.
\end{equation}
The evolution variable changes with \(\theta\) like
\begin{align} 
    \frac{dt}{d\theta} &= \frac{\alpha_s}{\pi} \frac{1}{\theta}.
\end{align}
and inserting this into \autoref{eqn: evolutionvariable_intermediate_condition1} gives
\begin{align}
     \frac{d\mathcal{P}}{dt dz} &= \frac{d\mathcal{P}}{d\theta dz} \frac{d\theta}{dt} \nonumber\\
    &= \left( \frac{\alpha_s}{\pi} \frac{1}{\theta} P(z) \right) \left( \frac{\pi}{\alpha_s} \theta \right) \nonumber \\
    &= P(z).
\end{align}
The evolution variable will therefore simplify the evolution equations by absorbing both the logarithms and constants which would have otherwise been explicitly written. 

Since the shower evolves down to the hadronization scale, we must have that the transverse momenta must be greater than \(Q_0 \sim 1 GeV >>\Lambda_{\text{QCD}}\). In the soft limit it can be written as,  
\begin{align}
    k_\perp = z(1-z)p_t \theta \sim z p_t \theta > Q_0.
\end{align}
With this evolution variable, the parton showers will evolve in angle from the jet radius \(R \sim 0.4\), down to the minimum angle, \(\theta_{\text{min}} = Q_0/p_t\), given by the hadronization scale\footnote{Splittings may still occur beyond this point as \(Q_0>>m_\pi\sim \Lambda_{\text{QCD}}\), and most of the partons hadronize into pions, but it is a convenient cutoff for our evolution.}. The limits on the evolution variable \(t\) is then,
\begin{equation}\label{eqn: evolution_variable__limits_vacuum}
    t_{\text{max}} = \frac{\alpha_s}{\pi} \ln \frac{p_tR}{Q_0} \quad, \qquad t_{\text{min}} = 0.
\end{equation}
Therefore, as we increase in values for the evolution variable \(t\) we will be reducing our emission angles, which leads to angular ordering being directly implemented into our showers.

The literature often writes the evolution variable as \(t \sim \ln \frac{Q}{Q_0}\), which is equivalent to ours when we recognize that \(Q=p_tR\) is the transverse jet scale. Since our choice of evolution variable \(t\) is dependent on the opening angle \(\theta\), angular ordering is native to our parton shower programs.

\subsubsection*{Vacuum splitting functions}
The Altarelli-Parisi splitting functions for vacuum branchings was calculated to leading order \(P_{ba}^{(0)}\) in \autoref{sec: derivation_splitting_functions_vacuum}. Higher-order corrections are available on the form 
\begin{equation}
    P_{ba} = P_{ba}^{(0)}+ \left(\frac{\alpha_s}{\pi}\right) P_{ba}^{(1)} + \left(\frac{\alpha_s}{\pi}\right)^2 P_{ba}^{(2)} + \cdots
\end{equation}
and next-to-leading order calculations are performed in~\cite{Gehrmann_2014}. Leading order is however sufficient for our treatment, and we will be using \autoref{eqn: vacuum_gg_splitting_function}, \autoref{eqn: vacuum_qg_splitting_function}, and \autoref{eqn: vacuum_qq_splitting_function} as our splitting functions in vacuum.

These splitting functions are however too complicated for some of the calculations we want to perform. It will therefore be useful to introduce a simplified splitting function for \(gg\) branchings in vacuum
\begin{align}\label{eqn: vacuum_gg_simple_splitting_function}
    P_{gg}^{\text{simple}}(z) &= \frac{C_A}{z(1-z)}
\end{align}
which will be used for some of the analytical calculations.

\subsection{The DGLAP equations}\label{sec: DGLAP_standard}
As mentioned in the discussion on observables, the inclusive parton distribution is governed by the Dokshitzer-Gribov-Lipatov-Altarelli-Parisi (DGLAP) evolution equations. They are written as a set of integro-differential equations that describe how the distribution of gluons and quarks evolve, respectively. While the general form of the equations presented is valid beyond leading logarithmic approximation, we will be focusing on the leading behavior, and setting \(\alpha_s\) to be constant, such that we are only interested in terms of the order \(t\) in our evolution variable (equivalent to order \(\alpha_s \ln(p_tR/Q_0)\)).

\subsubsection*{The full DGLAP equations}
The DGLAP equations have several conventions, so it is important to be consistent in the way it is presented. The DGLAP equations can be written in terms of the inclusive parton distribution \(f_i(x,t)\), as
\begin{align}
    \begin{split}
    \frac{\partial}{\partial t} f_g(x,t) &= \int_x^1 \frac{dz}{z} 2\,P_{gg}(z) f_g\left(\frac{x}{z},t\right) - \int_0^1 dz\, P_{gg}(z) f_g(x,t) \\
    &\quad + \int_x^1 \frac{dz}{z} P_{gq}(z) \, f_q\left(\frac{x}{z},t\right) - \int_0^1 dz P_{gq}(z) f_g(x,t)
    \end{split}\label{eqn: DGLAP_fg(x,t)}
\end{align}
\begin{align}
    \begin{split}
    \frac{\partial}{\partial t} f_q(x,t) &=  \int_x^1 \frac{dz}{z} P_{qq}(z) f_q\left(\frac{x}{z},t\right) - \int_0^1 dz\, P_{qq}(z) f_q(x,t) \\
    &\; + \int_0^1 dz P_{qg}(z) \frac{1}{z}\, f_g\left(\frac{x}{z},t\right)
    \end{split}\label{eqn: DGLAP_fq(x,t)}
\end{align}
where \(P(z)\) is the Altarelli-Parisi splitting functions, and \(f_{g/q}(x,t) = dN_{g/q}/dx\) is the inclusive parton distribution for gluons and quarks respectively. The factor \(2\) comes from the symmetry in the \(P_{gg}\) splitting function, as the emitted gluons can carry either the momentum \(z\) or \((1-z)\). These functions essentially represent how gluons and quarks can enter and leave a given volume element as illustrated in~\cite[p.166-168]{ellis_stirling_webber_1996}. In \autoref{app: DGLAP_derivation} we have constructed the DGLAP equation using generating functionals, in a manner similar to \cite{Probabilistic_picture}.

The evolution equations may also be written in terms of the energy distribution \(D_i(x,t) = x\, f_i(x,t)\). Adding a step-function \(\Theta(z>x)\) we can also gather the integrals to make it more compact, in which case \autoref{eqn: DGLAP_fg(x,t)} and \autoref{eqn: DGLAP_fq(x,t)} becomes, 
\begin{align}
\begin{split}
    \frac{\partial}{\partial t} D_g(x,t) &= \int_0^1 dz\,P_{gg}(z) \left[2\,D_g\left(\frac{x}{z},t\right)\, \Theta(z>x)-D_g(x,t)\right] \\
    &\quad + \int_0^1 dz P_{gq}(z) \left[ D_q\left(\frac{x}{z},t\right)\, \Theta(z>x) - D_g(x,t)\right]  
\end{split} \label{eqn: DGLAP_Dg(x,t)}
\end{align}
\begin{align}
    \begin{split}
    \frac{\partial}{\partial t} D_q(x,t) &= \int_0^1 dz\,P_{qq}(z) \left[D_q\left(\frac{x}{z},t\right)\, \Theta(z>x)-D_q(x,t)\right] \\
    &\quad + \int_x^1 dz P_{qg}(z) D_g\left(\frac{x}{z},t\right).
    \end{split} \label{eqn: DGLAP_Dq(x,t)}
\end{align}
Both these sets of equations will be used, depending on which observables we are interested in. The DGLAP presented here is to leading order, but higher-order expansions can also be written as is done in~\cite{Dasgupta_2015}.
\subsubsection*{The DGLAP equation for gluons}
We will frequently be focusing on cascades involving exclusively gluons, such that \(P_{gg}(z)\) is the only splitting we are considering. The evolution equations is then adjusted by simply disregarding the terms with quarks, such that \autoref{eqn: DGLAP_fg(x,t)} takes the form
\begin{equation}\label{eqn: DGLAP_gluons_f(x,t)}
    \frac{\partial}{\partial t} f_g(x,t) = \int_x^1 \frac{dz}{z} 2\,P_{gg}(z) f_g\left(\frac{x}{z},t\right) - \int_0^1 dz\, P_{gg}(z) f_g(x,t).
\end{equation}
It can be shown from the symmetry of the \(P_{gg}(z)\) splitting function that 
\begin{equation}\label{eqn: gg_Splitting_identitything}
    \int_0^1 dz \, z \, P_{gg}(z) = \frac{1}{2} \int_0^1 dz\, P_{gg}(z)
\end{equation}
and it is therefore possible to rewrite \autoref{eqn: DGLAP_gluons_f(x,t)}, to the form
\begin{equation}\label{eqn: DGLAP_energyflow}
    \frac{\partial}{\partial t} D(x,t) = \int_x^1 dz \,\tilde P_{gg}(z)\, D(x/z,t) - \int_0^1 dz\, z\, \tilde P_{gg}(z)\, D(x,t)
\end{equation}
where \(\tilde P_{gg}(z) \equiv 2\, P_{gg}(z) \). Both \autoref{eqn: DGLAP_gluons_f(x,t)} and \autoref{eqn: DGLAP_energyflow} will be used when discussing gluon showers.

\subsection{The Sudakov form factor in vacuum}
Now we will rewrite the DGLAP equation by introducing the Sudakov form factor, which will be important in developing the Monte-Carlo shower program in \autoref{cpt:num}. The Sudakov form factor is denoted as \(\Delta(t)\) and is defined using the evolution variable introduced in \autoref{eqn: evolution_variable_vacuum}. For gluon showers the Sudakov form factor is
\begin{align}\label{eqn: sudakov_form_factor_dasguptalike}
    \Delta (t) &= \exp\left(-t\, \int_{\epsilon}^{1-\epsilon}dz \, P_{gg}(z)\right)
\end{align}
with the derivatives
\begin{align}
    \frac{\partial}{\partial t} \Delta (t) &= \left(- \int_{\epsilon}^{1-\epsilon} dz\, P_{gg}(z) \right) \Delta(t) \nonumber\\
    \frac{\partial}{\partial t} \frac{1}{\Delta(t)}&= \int_{\epsilon}^{1-\epsilon} dz\, P_{gg}(z) \frac{1}{\Delta(t)}.
\end{align}
It is easy to rewrite the DGLAP equation an integral equation using the Sudakov form factor.
Starting by implementing the Sudakov into \autoref{eqn: DGLAP_gluons_f(x,t)},
\begin{align}
    \frac{\partial}{\partial t} f(x,t) + \int_{0}^{1} dz \, P_{gg}(z) f(x,t) &= \int_{x}^{1} \frac{dz}{z} \,2\, P_{gg}(z) f(x/z ,t) \nonumber \\
    \frac{\partial}{\partial t} f(x,t) + \Delta(t) f(x,t) \frac{\partial}{\partial t} \frac{1}{\Delta(t)} &= \int_{x}^{1} \frac{dz}{z} \,2\, P_{gg}(z) f(x/z ,t) \nonumber \\
    \frac{1}{\Delta(t)} \frac{\partial}{\partial t} f(x,t) - f(x,t) \frac{\partial}{\partial t} \frac{1}{\Delta(t)} &= \frac{1}{\Delta(t)} \int_{x}^{1} \frac{dz}{z} \,2\, P_{gg}(z) f(x/z ,t) \nonumber \\
    \frac{\partial }{\partial t} \frac{f(x,t)}{\Delta (t)} &= \frac{1}{\Delta(t)} \int_{x}^{1} \frac{dz}{z} \,2\, P_{gg}(z) \, f(x/z, t).
\end{align}
The Sudakov form factor has now been implemented properly into the evolution equation. Continuing we wish to rewrite from a differential to an integral form,
\begin{align}
    \int_{t_0}^t dt' \frac{\partial}{\partial t} \frac{f(x,t')}{\Delta(t')} = &\int_{t_0}^{t} \frac{dt'}{t'} \frac{1}{\Delta(t')} \int_{x}^{1} dz \,2\, P_{gg}(z) \frac{1}{z} f(x/z, t) \nonumber\\
    \frac{f(x,t)}{\Delta(t)} - \frac{f(x,t_0)}{\Delta(t_0)} = &\int_{t_0}^{t} \frac{dt'}{t'} \frac{1}{\Delta(t')} \int_{x}^{1} dz \,2\, P_{gg}(z) \frac{1}{z} f(x/z, t) \nonumber\\
    \frac{f(x,t)}{\Delta(t)} - f(x,t_0) &= \int_{t_0}^{t} \frac{dt'}{t'} \frac{1}{\Delta(t')} \int_{x}^{1} dz  \,2\, P_{gg}(z) \frac{1}{z} f(x/z, t)
\end{align}
and we end up with our desired evolution equation on integral form
\begin{equation}\label{eqn: DGLAP_evolutioneq_unregularized_integral_ellis} %Ellis 5.49
    f(x,t) = \Delta(t) f(x,t_0) + \int_{t_0}^{t} \frac{dt'}{t'} \frac{\Delta(t)}{\Delta(t')} \int_{x}^{1} \frac{dz}{z} \,2\, P_{gg}(z) \, f(x/z, t').
\end{equation}
The physical interpretation of this equation comes from considering the Sudakov form factor as a no-branching probability. This means that the first term on the right-hand side of \autoref{eqn: DGLAP_evolutioneq_unregularized_integral_ellis} is the probability of no branching between \(t_0\) and \(t\) (since \(\Delta(t_0)=1\)). The second term then represents that some branching has occurred at time \(t'\), followed by no branchings between \(t'\) and \(t\). The Sudakov form factor is effectively a way of resumming all orders of the DGLAP equation, by absorbing the divergences into the Sudakov form factor~\cite{schwartz2014quantum}. %By iterating over the full evolution, an arbitrary amount of emissions may occur, which is equivalent to performing a resummation of all orders.

If we are working with both quarks and gluons then the Sudakov form factor will be slightly different, as several different branchings may occur in a given interval. The Sudakovs for showers with both gluons and quarks will be respectively
\begin{align}
    \Delta_{g}(t) &= \exp\left(-t \int_\epsilon^{1-\epsilon} \left[P_{gg}(z)+P_{qg}(z) \right] \, dz \right)  \label{eqn: sudakov_vacuum_gluons}  \\
    \Delta_{q}(t) &= \exp\left(-t \int_\epsilon^{1-\epsilon} P_{qq}(z) \, dz \right). \label{eqn: sudakov_vacuum_quarks}
\end{align}

\subsection{Analytical solution of the DGLAP equation}\label{sec: solution_DGLAP}
We will now solve the DGLAP equation following the method outlined in~\cite{Energy_flow_medium_cascade_2016}, for gluons in vacuum. The starting point is \autoref{eqn: DGLAP_energyflow}, which is written with \(\tilde P_{gg}(z) \equiv 2\, P_{gg}(z)\). The Mellin transform and its inverse are defined as
\begin{equation}\label{eqn: mellin_transforms}
    \tilde{D}(\nu, t) = \int_0^1 dx\, x^{\nu-1} \, D(x,t) \quad, \quad D(x,t) = \int_{c-i\infty}^{c+i\infty} \frac{d\nu}{2\pi i}\, x^{-\nu} \, \tilde{D}(\nu, t).
\end{equation}
Taking the Mellin transform of \autoref{eqn: DGLAP_energyflow} and then changing the integration limits \(\int_0^1 dx \int_x^1 dz \rightarrow \int_0^1 dz \int_0^x dx\), and performing a change of variable \(\xi = x/z \; \Rightarrow \; dx=z\, d\xi\) we obtain
\begin{align}
    \frac{\partial}{\partial t} \int_0^1 dx\,x^{\nu-1} D(x,t) &= \int_0^1 dx\,x^{\nu-1} \int_x^1 dz\, \tilde{P}_{gg}(z)\, D(x/z,t) - \int_0^1 dx\,x^{\nu-1}\, \int_0^1 dz\, z\,\tilde{P}_{gg}(z)\, D(x,t) 
    \nonumber \\
    \frac{\partial}{\partial t} \tilde D(\nu,t) &= \int_0^1 dz\, \tilde{P}_{gg}(z)\, \int_0^1 (z\, d\xi)\, (z\, \xi)^{\nu-1} D(\xi,t) - \int_0^1 dz\, z\, \tilde{P}_{gg}(z)\, \tilde D(\nu,t) \nonumber \\
    \frac{\partial}{\partial t} \tilde D(\nu,t) &= -\int_0^1 dz\,\tilde{P}_{gg}(z) \left(z-z^\nu \right) \tilde D(\nu,t).
\end{align}
Inserting the simplified splitting function of \autoref{eqn: vacuum_gg_simple_splitting_function}, we can write the evolution equation in Mellin space as
\begin{align}
    \frac{\partial}{\partial t} \tilde D(\nu,t) &= -2C_A \int_0^1 dz \frac{1-z^{\nu-1}}{(1-z)} \tilde D(\nu,t). \label{eqn: my_mellin_transform}
\end{align}
Introducing the Digamma function
\begin{equation}\label{eqn: digamma_function}
    \psi(\nu) = \int_0^1 \frac{1-z^{\nu-1}}{1-z}dz - \gamma_E,
\end{equation}
where \(\gamma_E\) is the \emph{Euler-Mascheroni constant}, we can write \autoref{eqn: my_mellin_transform} as a simple differential equation
\begin{equation}
    \frac{\partial}{\partial t} \tilde D(\nu,t) = -2C_A (\psi(\nu)+\gamma_E) \tilde D(\nu,t)
\end{equation}
the solution is easily obtained with the initial condition \(\tilde D(\nu,0) = 1\)
\begin{equation}\label{eqn: DGLAP_solution_mellinspace}
    \tilde D(\nu,t) = \exp \left[-2C_A (\psi(\nu)+\gamma_E)t\right].
\end{equation}
Now that the solution is obtained, we just have to transform back with the inverse Mellin transform
\begin{align}\label{eqn: DGLAP_solution_mellin_intermediate}
    D(x,t) &= \int_{c-i\infty}^{c+i\infty}\frac{d\nu}{2\pi i} x^{-\nu} \exp \left[-2C_A (\psi(\nu)+\gamma_E)t\right] \nonumber \\
    &= \int_{c-i\infty}^{c+i\infty}\frac{d\nu}{2\pi i} \exp\left[ \ln 1/x^\nu\right] \exp \left[-2C_A (\psi(\nu)+\gamma_E)t\right] \nonumber \\
    &= \int_{c-i\infty}^{c+i\infty}\frac{d\nu}{2\pi i} \exp \left[-2C_A (\psi(\nu)+\gamma_E]t + \nu \ln \frac{1}{x} \right]. 
\end{align}
A solution to the equation an be obtained by utilizing the saddle-point approximation. Setting the argument of the exponential as \(f(\nu) \equiv -2C_A (\psi(\nu)+\gamma_E)t + \nu \ln \frac{1}{x} \), then in a saddle point \(\nu_s\) we should have
\begin{equation}
    f(\nu) \approx f(\nu_S) + \frac{1}{2} f''(\nu_S)(\nu-\nu_S)^2.
\end{equation}
\(f(\nu)\) has a minimum value when 
\begin{align}\label{eqn: DGLAP_mellin_solution_minimum}
    \psi'(\nu_s) &= \frac{1}{2C_A}\,\frac{\ln (1/x)}{t}
\end{align}
which gives us, in the small \(x\) limit where \(\ln (1/x) >> t\), 
\begin{align}
    \psi(\nu) \approx -\nu^{-1} \quad &\Rightarrow \quad \psi'(\nu) \approx \nu^{-2}.
\end{align}
and the value \(\nu_s\) of the saddle point can therefore be found to be
\begin{align}
    \nu_s &= \left(\frac{2C_A\,t}{\ln(1/x)}\right)^{1/2}.
\end{align}
We can then write \autoref{eqn: DGLAP_solution_mellin_intermediate} by using the saddle point-approximation
\begin{align}
    D(x,t) &\approx \exp \left(f(\nu_S) \right) \int_{c-i\infty}^{c+i\infty}\frac{d\nu}{2\pi i} \exp \left(\frac{1}{2} f''(\nu_s)(\nu-\nu_s)^2 \right).
\end{align}
Performing a change of variable \(\nu = i b\) to remove the complex numbers in the integral
\begin{align}
    D(x,t) &\approx \exp \left(f(\nu_S) \right) \int_{-\infty}^{\infty}\frac{db}{2\pi} \exp \left(-\frac{1}{2} f''(\nu_s)(b+i\nu_s)^2 \right) 
\end{align}
and solving using Mathematica,
\begin{equation}
    \int_{-\infty}^{\infty}\frac{db}{2\pi} \exp \left(-\frac{1}{2} f''(\nu_s)(b+i\nu_s)^2 \right) = \frac{1}{2\sqrt{\pi/2}\sqrt{f''(\nu_s)}}.
\end{equation}
Seeing that \(f''(\nu_S) = 4C_A\,t\, \nu_S^{-3}\), we write our solution as
\begin{align}\label{eqn: DGLAP_solution_energyflowmedium}
    D(x,t) &= exp \left(f(\nu_S) \right)  \frac{1}{2\sqrt{(\pi/2) \, f''(\nu_S)}}  \nonumber\\
    D(x,t) &= exp \left(2C_A \nu_S^{-1}-2C_A\gamma_E t + \nu_S \ln \frac{1}{x} \right)  \frac{1}{2} \frac{1}{\sqrt{2\pi C_A\,t\, \nu_S^{-3}}}  \nonumber\\
    %D(x,t) &= exp \left(2\sqrt{2C_A} \sqrt{t \ln \frac{1}{x}}t - 2C_A\gamma_E \,t \right)  \frac{1}{2} \frac{1}{\sqrt{2 \pi C_A\,t} } \left(\frac{2C_A\,t}{\ln(1/x)}\right)^{3/4} \nonumber\\
    D(x,t) &= exp \left(2\sqrt{2C_A} \sqrt{t \ln \frac{1}{x}} - 2C_A\gamma_E \,t \right)  \frac{1}{2} \left(\frac{2C_A\,t}{\pi^2 \ln^3(1/x)}\right)^{1/4}.
\end{align}
\autoref{eqn: DGLAP_solution_energyflowmedium} is therefore a solution of the DGLAP equation. Note that the factor \(\sqrt{2C_A}\) comes from the symmetry factor and color factor of the \(P_{gg}(z)\) splitting function. It is valid in the small \(x\)-limit such that \(\ln \frac{1}{x} >> t\). In the double logarithmic limit, where we focus on the leading \(\ln \frac{1}{x}\) behavior, the solution can be written as
\begin{align}\label{eqn: DGLAP_DLL_solution}
    D_{\textit{DLL}}(x,t) &\approx exp \left(2\sqrt{2C_A} \sqrt{t \ln \frac{1}{x}} \right) \nonumber \\
    %&\approx exp \left(2\sqrt{2C_A} \ln \frac{1}{x} \sqrt{\frac{t}{\ln \frac{1}{x}}} \right) \nonumber \\
    &\approx \left(\frac{1}{x}\right)^{ 2\sqrt{2C_A} \sqrt{\frac{t}{\ln 1/x}}}.
\end{align}

\section{Formalism of Parton Branching in Medium}
This section will explore parton showers developing in a dense QCD medium. Most of the notation will be given in the same manner as~\cite{Energy_flow_medium_cascade_2016}. The new medium showers will be described by a in-medium kinetic rate equation which has a similar structure to the DGLAP equation. After introducing the differences between vacuum and medium cascades, we will solve the evolution equation for gluons. Our treatment will begin by disregarding the broadening that can occur between splittings, and rather focus on how the cascade changes due to the large amount of soft gluon emission. At the end of the section, we will discuss how to account for the broadening of partons due to exchange of momenta with the medium.

We will be working exclusively with gluons throughout this section. Subsequent discussions on medium showers will also be purely gluonic.

\subsection{Properties of medium cascades}\label{sec: BDMPS_properties}
\subsubsection*{Differences between vacuum and medium cascades}
Before introducing the in-medium kinetic rate equation, a discussion is required for covering the main differences between medium and vacuum showers. In the vacuum discussion, the first point of interest was the evolution variable which was a dimensionless quantity used to conveniently write the evolution equations. For the medium showers this will be replaced by the characteristic time \(t_*\), which has dimensions equal to the actual time \([GeV^{-1}]\) and is defined in \autoref{eqn: characteristic_time}. The characteristic time, or stopping time, is the time it takes for a gluon of energy \(\omega\) to radiate most of its energy into soft gluons
\begin{equation}\label{eqn: characteristic_time}
    t_* \equiv \frac{1}{\bar \alpha} t_{\text{br}}(E) = \frac{\pi}{\alpha_s N_C} \sqrt{\frac{E}{\hat q}}
\end{equation}
here \(t_{\text{br}}(E)\) is the typical time, or branching time, introduced in \autoref{eqn: branching_time}, which is the time it takes a gluon of energy E to branch into two gluons, and \(\hat q\) is the \emph{jet-quenching parameter}. Since we are exclusively working with gluons, we can set \(C_A=N_C\), and define \(\bar \alpha \equiv \alpha_s N_C / \pi\) and \(\hat{q} = \hat{\bar q} \,C_A\), such that all color factors are absorbed into the branching time. 

An important feature of the medium cascades is that the branching rate increases along the cascade, in contrast to the vacuum cascades, where the branching rate is constant along the cascade. This is apparent when looking at the characteristic time; as the energy \(\omega\) of a given gluon decrease, the expected time for branching also decreases. This means that the branchings are accelerating, and it takes a finite time to transport a finite amount of energy from the leading particle to soft gluons. Energy is therefore effectively transported towards large angles, which contrasts the strong angular ordering of QCD cascades in vacuum. Both the increased branching rate and transport of energy towards large angles are natural consequences of medium induced soft gluon emissions. The spectrum of these radiated gluons is on the form
\begin{equation}\label{eqn: BDMPS_spectrum}
    \omega \frac{dI}{d\omega} \approx \bar \alpha \sqrt{\frac{\omega_c}{\omega}}
\end{equation}
where \(\omega_c\) is the energy which gives one branching in the length of the traversed medium \(t_{\text{br}}(\omega_c) \sim L\), and can therefore be written as \(\omega_c = \hat q L^2\). The spectrum in \autoref{eqn: BDMPS_spectrum} is called the BDMPS-Z spectrum, named after Baier-Dokshitzer-Mueller-Peigne-Schiff and Zakharov. The first factor \(\bar \alpha\) represents the standard bremsstrahlung spectrum for radiation by the parent gluon, while the second factor \(\sqrt{\omega_c/\omega}\) is a correction factor. This factor needs to be cut off for a minimum frequency at which radiation is primarily produced by incoherent collisions. This cut-off \(\omega_{\text{BH}}\) is where the branching time is of the order of the mean free path, \(t_{\text{br}} \sim \ell\). The mechanisms and approximations required for large amounts of soft gluon emissions require the formation time to be much larger than the mean free path \(\ell\), but smaller than the size of the medium \(L\), this gives us boundaries for the energy of the emitted gluons \(\omega_{\text{BH}} << \omega < \omega_c\)~\cite{medium_induced_gluon_branching}.

Since this section is concerned with partons traversing a medium, it could also be insightful to rewrite the BDMPS-Z spectrum of \autoref{eqn: BDMPS_spectrum} in terms of \(dL\), in order to indicate the branching rate when traversing a given length of medium. Simply inserting \(\omega_c = \hat{q}L^2\), we can write
\begin{align}\label{eqn: BDMPS_branching_rate}
    \frac{dI}{d\omega dL} &= \frac{\bar \alpha}{\omega} \sqrt{\frac{\hat{q}}{\omega}}.
\end{align}

\subsubsection*{Scaling behavior}
Another feature of the medium cascades is that they exhibit a scaling behavior in the number of partons occupying small values of \(x\). This scaling stems from the solution of the BDMPS-Z spectrum which we will solve shortly, and is on the form
\begin{equation}\label{eqn: BDMPS_scaling_behavior}
    D(x,t) \approx \frac{t}{t_*\sqrt{x}} \exp \left[-\pi \left(\frac{t}{t_*}\right)^2\right].
\end{equation}
The scaling manifest itself when \(t > t_*\), which means that most of the energy has been radiated into soft gluons \(x<0.1\). After this, the number of partons occupying a given element \(\delta x\) decreases in a uniform and shape-conserving way. A natural interpretation of this phenomenon would be a constant flow of energy towards small values of \(x\), which can be related to the existence of a stationary solution of the energy distribution~\cite{Energy_flow_medium_cascade_2016}. This scaling will be apparent later when we will create a Monte-Carlo program for simulating parton showers in medium, for different values of \(t\). 

\subsubsection*{Medium splitting functions}\label{sec: medium_splitting_functions}
The splitting functions responsible for parton branchings in medium are somewhat different from their vacuum counterparts. From our discussion on the BDMPS-Z spectrum we would expect the splitting functions to be similar to the branching rate given by \autoref{eqn: BDMPS_branching_rate}. Since we have absorbed all color factors into the branching time, the \(gg\) splitting function given by ~\cite{Universal_quark_gluon_ratio_in_medium-induced_parton_cascade} can be written as,
\begin{align}
    \mathcal{K}_{gg}(z) &= \frac{1}{2}\,2\, \frac{\left[1-z(1-z) \right]^{5/2}}{\left[z(1-z)\right]^{3/2}}. \label{eqn: medium_splittingfunctions_gg}
\end{align}
The factor \(1/2\) come from the terms accounting for medium induced radiation, while the factor \(2\) comes from the symmetry of the \(gg\) splitting. Comparing with the vacuum counterpart we now have an additional term inside a square root. This appears due to partons interacting with the medium, leading to soft gluon emissions and making the medium splitting function more divergent. The parton distributions are therefore softened, or quenched, and is the only measurable consequence of the quark energy-loss in the medium~\cite{Wang_2001_Multiple_Parton_Scattering}.

While this splitting function might seem quite different from its vacuum counterpart at first glance, the simple vacuum splittings functions can be easily identified such that \autoref{eqn: medium_splittingfunctions_gg} can be written in terms the vacuum splitting function \(P_{gg}(z)\). This form also accentuate the similarities of \autoref{eqn: BDMPS_branching_rate} and the splitting function
\begin{align}\label{eqn: vacuumtomedium_ggg_splitting_relation}
    C_A\,\mathcal{K}_{gg}(z) &= C_A \, \frac{[1-z(1-z)]^2}{z(1-z)} \, \sqrt{\frac{(1-z)+ z^2 }{z(1-z)}} \nonumber \\
    &= P_{gg}(z) \,\sqrt{\frac{1-z(1-z)}{z(1-z)}}
\end{align}
The full splitting function has now been introduced. However, it will be sufficient - and more convenient - to work with a simplified splitting function which we will write as
\begin{equation}\label{eqn: ggg_medium_reduced_kernel}
    \mathcal{K}(z) = \frac{1}{\left( z(1-z)\right)^{3/2}}.
\end{equation}
For the remainder of the chapter, \autoref{eqn: ggg_medium_reduced_kernel} will be used as our \(gg\) splitting function, and we will call it the reduced kernel.

\subsection{The in-medium kinetic rate equation}\label{sec: BDMPS_theory}
\subsubsection*{The full in-medium kinetic rate equations}
The necessary ingredients for constructing the in-medium kinetic rate equations have now been introduced. It is expected that the form of the evolution equations in medium, will be very similar to the vacuum version. For a derivation using generating functionals see~\cite{Probabilistic_picture}, here we simply quote the results following the conventions of~\cite{Energy_flow_medium_cascade_2016}. An important feature to note is that the in-medium kinetic rate equation is written in terms of the actual time \(t\) and characteristic time \(t_*\) (defined in \autoref{eqn: characteristic_time}). This means that angular ordering is not strictly imposed in these showers, which is to be expected as soft gluon emission at large angles occurs frequently due to medium interactions. For now we will write the characteristic time as a function of the energy \(t_*(x) = t_* \sqrt{x}\). This gives an effective time scale for the branching of a gluon carrying a fraction \(x\) of the initial energy, which is one of the properties of the medium cascade. The evolution equations can be written
\begin{align}\label{eqn: BDMPS_gluons}
\begin{split}
    \frac{\partial}{\partial t} D_g(x,t) &= \int_x^1 dz\, \frac{1}{t_*(x/z)} \mathcal{K}_{gg}(z)\, D_g\left(\frac{x}{z}, t\right) - \frac{1}{2t_*(x)} \int_0^1 dz\, \mathcal{K}_{gg}(z)\, D_g \left(x,t\right) \\
    &\quad + \int_x^1 dz\, \frac{1}{t_*(x/z)} \mathcal{K}_{gq}(z)\, D_g\left(\frac{x}{z}, t\right) - \frac{1}{t_*(x)} \int_0^1 dz\, \mathcal{K}_{gq}(z)\, D_g \left(x,t\right)
\end{split}
\end{align}
\begin{align}\label{eqn: BDMPS_quarks}
\begin{split}
    \frac{\partial}{\partial t} D_q(x,t) &= \int_x^1 dz\, \frac{1}{t_*(x/z)} \mathcal{K}_{qq}(z)\, D_q\left(\frac{x}{z}, t\right) -\frac{1}{t_*(x)} \int_0^1 dz\, \mathcal{K}_{qq}(z)\, D_q \left(x,t\right) \\
    &\quad + \int_x^1 dz\, \frac{1}{t_*(x/z)} \mathcal{K}_{qg}(z)\, D_g\left(\frac{x}{z}, t\right).
\end{split}
\end{align}
The main difference between the in-medium kinetic rate equations, and the DGLAP equations should then be apparent. If we replace \(\mathcal{K}_{ba}(z) \rightarrow \tilde{P}_{ba}(z)\) and \(t_*(x)\rightarrow 1\), the in-medium kinetic rate equations reduces to the DGLAP equations.

\subsubsection*{The in-medium kinetic rate equation for gluons}
We will be primarily focusing on cascades consisting exclusively of gluons. Since the splitting function \(\mathcal{K}_{gg}(z)\) is perfectly symmetrical, we can again use that
\begin{equation}
    \frac{1}{2} \int_\epsilon^{1-\epsilon} dz \mathcal{K}_{gg}(z) = \int_\epsilon^{1-\epsilon} dz\, z\, \mathcal{K}_{gg}(z)
\end{equation}
and \autoref{eqn: BDMPS_gluons} can be written, 
\begin{equation}\label{eqn: BDMPS_2.8_Blaizot}
    \frac{\partial}{\partial t} D(x,t) = \int_x^1 dz\,\frac{1}{t_*(x/z)} \mathcal{K}_{gg}(z)\, D\left(\frac{x}{z},t\right) - \frac{1}{t_*(x)} \int_0^1 dz\,\mathcal{K}_{gg}(z) \,z\, D(x,t)
\end{equation}
The evolution equation for gluons can also be written by noting that the splitting kernel is independent of time. The evolution equation can therefore be simplified by introducing a new variable \(\tau\) which accounts for the energy dependence and time scale of the branchings
\begin{equation}\label{eqn: medium_tau_definiton}
    \tau = \frac{t}{t_*}= \bar \alpha \sqrt{\frac{\hat{q}}{E}} t
\end{equation}
using this new variable the evolution equation for gluons can be written as
\begin{equation}\label{eqn: BDMPS_solution_startingpoint}
    \frac{\partial}{\partial \tau} D(x, \tau) = \int_x^1 dz \,\mathcal{K}(z) \sqrt{\frac{z}{x}} D\left(\frac{x}{z}, \tau\right) - \int_0^1 dz \,\mathcal{K}(z) \frac{z}{\sqrt{x}} D(x,t).
\end{equation}

\subsection{The Sudakov form factor in medium}\label{sec: medium_sudakov}
We will now start with \autoref{eqn: BDMPS_2.8_Blaizot}, and rewrite it in terms of a Sudakov form factor
\begin{align}\label{eqn: BDMPS_sudakov}
    \Delta (t) &= \exp \left( -\frac{t}{t_*(x)} \int_0^1 \, dz\, z \mathcal{K}(z) \right)
\end{align}
with the derivatives, 
\begin{align}
    \frac{\partial}{\partial t} \Delta (t) &= - \frac{1}{t_*(x)} \int_0^1 \, dz\, z \mathcal{K}(z)  \Delta(t) \nonumber\\
    \frac{\partial}{\partial t} \frac{1}{\Delta (t)} &= \frac{1}{t_*(x)} \int_0^1 \, dz\, z \mathcal{K}(z)  \frac{1}{\Delta(t) }
\end{align}
Starting by rewriting \autoref{eqn: BDMPS_2.8_Blaizot}
\begin{align}
    \frac{\partial}{\partial t} D(x,t) + \frac{D\left(x,t\right)}{t_*(x)} \int_0^1 dz\, z\, \mathcal{K}(z) &= \int_x^1 dz\, \mathcal{K}(z)\, \frac{D\left(\frac{x}{z}, t\right)}{t_*(x/z)} \nonumber\\
    \frac{\partial}{\partial t} D(x,t) + D(x,t) \frac{\partial}{\partial t} \frac{1}{\Delta(t)} &= \int_x^1 dz\, \mathcal{K}(z)\, \frac{D\left(\frac{x}{z}, t\right)}{t_*(x/z)} \nonumber\\
    \Delta(t) \frac{\partial}{\partial t} \left( \frac{D(x,t)}{\Delta(t)} \right) &= \int_x^1 dz\, \mathcal{K}(z)\, \frac{D\left(\frac{x}{z}, t\right)}{t_*(x/z)} \nonumber\\
    \frac{\partial}{\partial t} \left( \frac{D(x,t)}{\Delta(t)} \right) &= \frac{1}{\Delta(t)} \, \int_x^1 dz\, \mathcal{K}(z)\, \frac{D\left(\frac{x}{z}, t\right)}{t_*(x/z)} 
\end{align}
and integrating out the t integral, 
\begin{align}
    \frac{D(x,t)}{\Delta(t)} - \frac{D(x,t_0)}{\Delta(t_0)} &= \int_{t_0}^t \frac{dt'}{\Delta(t')} \, \int_x^1 dz\, \mathcal{K}(z)\, \frac{D\left(\frac{x}{z}, t'\right)}{t_*(x/z)} \nonumber\\
    D(x,t) &= D(x,t_0)\, \frac{\Delta(t)}{\Delta(t_0)} + \int_{t_0}^t dt' \, \frac{\Delta(t)}{\Delta(t')} \, \int_x^1 dz\, \mathcal{K}(z)\, \frac{D\left(\frac{x}{z}, t'\right)}{t_*(x/z)}.
\end{align}
If we now consider the initial time \(t_0 = 0\), then \(\Delta(t_0) = 1\) and we obtain an equation which is the medium equivalent of \autoref{eqn: DGLAP_evolutioneq_unregularized_integral_ellis}
\begin{equation}
    D(x,t) = D(x,t_0)\, \Delta(t) + \int_{t_0}^t dt' \, \frac{\Delta(t)}{\Delta(t')} \, \int_x^1 dz\, \mathcal{K}(z)\, \frac{D\left(\frac{x}{z}, t'\right)}{t_*(x/z)}.
\end{equation}

\subsection{Analytical solution of the in-medium kinetic rate equation}\label{sec: BDMPS_solution}
We will now solve the medium evolution equation for gluons, by closely following the method outlined in~\cite{Energy_flow_medium_cascade_2016}.
The starting point for solving the medium evolution equation is \autoref{eqn: BDMPS_solution_startingpoint}, where \(\mathcal{K}_{gg}(z)\) is the reduced kernel given in \autoref{eqn: ggg_medium_reduced_kernel}, and \(\tau \) is defined as in \autoref{eqn: medium_tau_definiton}.

The solution presented here is valid for values \(x_c > 1\), which corresponds to a large medium \(L>t_*\). This is a valid comparison with the Monte-Carlo program which is developed in \autoref{cpt:num}, as it assumes the shower is constantly evolving in a medium. 

\textbf{The first step} is to perform a change of variable such that \(\xi = \frac{x}{z}\) in the gain term and \(\xi = xz\) in the loss term
\begin{align}\label{eqn: BDMPS_solution_gainterm_changeofvariable}
    \mathbb{G} &= \int_x^1 dz \mathcal{K}(z) \sqrt{\frac{z}{x}} D\left(\frac{x}{z}, \tau\right) \qquad,\quad \xi = \frac{x}{z} \nonumber \\
    &= \int_1^x d\xi \left(-\frac{x}{\xi^2}\right) \mathcal{K}\left(\frac{x}{\xi}\right) \sqrt{\frac{1}{\xi}} D(\xi, \tau) \nonumber \\
    &= \int_x^1 d\xi \,\frac{x}{\xi^{5/2}} \mathcal{K}\left(\frac{x}{\xi}\right) D(\xi, \tau)
\end{align}
\begin{align}\label{eqn: BDMPS_solution_lossterm_changeofvariable}
    \mathbb{L} &= - \int_0^1 dz \mathcal{K}(z) \frac{z}{\sqrt{x}} D(x,\tau)\qquad , \quad \xi = xz \nonumber \\
    &= - \int_0^x d\xi \left(\frac{1}{x}\right) \mathcal{K}\left(\frac{\xi}{x}\right) \frac{\xi}{x^(3/2)} D(x,\tau) \nonumber \\
    &= - \int_0^x d\xi \,\frac{\xi}{x^{5/2}} \mathcal{K}\left(\frac{\xi}{x}\right) D(x,\tau)
\end{align}
in these equations a common splitting function can be identified as, 
\begin{align}\label{eqn: BDMPS_solution_splittingfunction_xivariable}
    P(x,\xi) &= \frac{x}{\xi^{5/2}} \mathcal{K}\left(\frac{x}{\xi}\right) \nonumber \\
    %&= \frac{x}{\xi^{5/2}} \frac{1}{\left[\frac{x}{\xi}(1-\frac{x}{\xi})\right]^{3/2}} \nonumber \\
    %&= \frac{x}{\xi^{5/2}} \frac{1}{ \frac{1}{\xi^3} \left[x(\xi-x)\right]^{3/2}} \nonumber \\
    &= \sqrt{\frac{\xi}{x}} \frac{1}{(\xi-x)^{3/2}}
\end{align}
and \autoref{eqn: BDMPS_solution_startingpoint} can therefore be written as 
\begin{equation}\label{eqn: BDMPS_solution_evoleqn_with_xisplitfunc}
    \partial_\tau D(x,\tau) = \int_x^1 d\xi \,P(x,\xi) D(\xi, \tau) - \int_0^x d\xi \,P(\xi,x) D(x,\tau).
\end{equation}
Note that \(P(x,\xi) \neq P(\xi, x)\). 
Now that the gain and loss terms are written in a convenient and symmetrical way, \textbf{the second step} is to deal with the integral of the loss term
\begin{equation}
    \int_0^x d\xi \, \sqrt{\frac{1}{\xi}} \frac{1}{(x-\xi+\epsilon)^{3/2}} = \frac{1}{\sqrt{\epsilon}} \frac{2x}{x+\epsilon} \approx \frac{2}{\sqrt \epsilon} - \frac{2\sqrt{\epsilon}}{x} + \mathcal{O}(\epsilon^{3/2}).
\end{equation}
In the limit \(\epsilon \rightarrow 0\), the first term is divergent, and all subleading terms vanish for any finite value of \(x\). Therefore, the sole purpose of the loss term (in these variables) is to remove the singularity of the gain term.  We can therefore replace the integral in the loss term with the following 
\begin{equation}
    \mathbb{L} = - D(x,\tau) \, \int_0^\infty \frac{dz}{z^{3/2}}.
\end{equation}
\textbf{Step three} we introduce a re-scaling of the distribution \(F(y, \tau ) = \sqrt{x} D(x,\tau)\) , where \(y = 1-x\), by multiplying everything by \(\sqrt{x}\) and inserting the new loss term, \autoref{eqn: BDMPS_solution_evoleqn_with_xisplitfunc} becomes
\begin{align}
    \partial_\tau \sqrt{x} D(x,\tau) &= \int_x^1 d\xi \, \frac{1}{(\xi-x)^{3/2}} \sqrt{\xi} \,D(\xi, \tau) - \sqrt{x}\, D(x,\tau) \, \int_0^\infty \frac{dz}{z^{3/2}}, \nonumber \\
    \begin{split}
        \partial_\tau F(y,\tau) &= \int_{1-y}^1 d\xi \, \frac{1}{[\xi-(1-y)]^{3/2}} \sqrt{\xi} \,D(\xi, \tau) \\
        &\quad - \sqrt{(1-y)}\, D((1-y),\tau) \, \int_0^\infty \frac{dz}{z^{3/2}}
    \end{split}.
\end{align}
Then performing a change of variable \(\tilde{\xi} = 1- \xi\), and using \(F(y, \tau ) = \sqrt{1-y} \,D(1-y,\tau)\)
\begin{align}\label{eqn: BDMPS_solution_evoleqn_Laplace_ready}
    \partial_\tau F(y,\tau) &= \int_0^{y} d\tilde{\xi} \, \frac{1}{(y-\tilde{\xi})^{3/2}} \sqrt{1-\tilde{\xi}} \,D(1-\tilde{\xi}, \tau) - F(y,\tau) \, \int_0^\infty \frac{dz}{z^{3/2}}, \nonumber \\
    &= \int_0^{y} d\tilde{\xi} \, \frac{1}{(y-\tilde{\xi})^{3/2}} F(\tilde{\xi}, \tau) - F(y,\tau) \, \int_0^\infty \frac{dz}{z^{3/2}}.
\end{align}
\textbf{Step four} is to extend the limits of the domain for \(F(y,\tau)\) from \(y\in[0,1] \rightarrow y \in [0, \infty]\), and Laplace transform our evolution equation. Defining the Laplace transform as
\begin{equation}\label{eqn: BDMPS_solution_Laplace_definition}
    \tilde{F}(\nu, \tau) = \int_0^\infty dy \, e^{-\nu y}\, F(y,\tau) 
\end{equation}
and performing the Laplace transform on \autoref{eqn: BDMPS_solution_evoleqn_Laplace_ready} gives
\begin{align}\label{eqn: BDMPS_solution_laplace_step1}
    \partial_t \tilde{F}(\nu,\tau) &= \int_0^\infty dy\, e^{-\nu y}\,  \int_0^{y} d\tilde{\xi} \, \frac{1}{(y-\tilde{\xi})^{3/2}} F(\tilde{\xi}, \tau) - \int_0^\infty dy \, e^{-\nu y}\, F(y,\tau) \, \int_0^\infty \frac{dz}{z^{3/2}} \nonumber \\
    &= \int_0^\infty dy\, \int_0^{y} d\tilde{\xi} \, e^{-\nu y} \frac{1}{(y-\tilde{\xi})^{3/2}} F(\tilde{\xi}, \tau) - \tilde{F}(y,\tau) \, \int_0^\infty \frac{dz}{z^{3/2}}.
\end{align}
Since the loss term had one \(y\) dependence, the Laplace transform went very smoothly. When dealing with the gain term it is necessary to make some changes to the integration boundaries \(\int_0^\infty dy\, \int_0^y d\tilde{\xi} \,\rightarrow\, \int_{\tilde{\xi}}^\infty dy\, \int_0^\infty d\tilde{\xi}\), and then introduce another change of variable \(z = y-\tilde{\xi}\)
\begin{align}\label{eqn: BDMPS_solution_gainterm_laplace}
    \mathbb{G} &= \int_0^{\infty} d\tilde{\xi} \,F(\tilde{\xi}, \tau) \int_{\tilde{\xi}} ^\infty dy\,  e^{-\nu y} \frac{1}{(y-\tilde{\xi})^{3/2}} \nonumber \\
    &= \int_0^{\infty} d\tilde{\xi} \,F(\tilde{\xi}, \tau) \int_0^\infty dz\, \frac{e^{-\nu (z+\tilde{\xi})}}{z^{3/2}} \nonumber \\
    &= \int_0^{\infty} d\tilde{\xi} e^{-\nu \tilde{\xi}} \,F(\tilde{\xi}, \tau) \int_0^\infty dz\, \frac{e^{-\nu z}}{z^{3/2}} \nonumber \\
    &= \tilde{F}(\nu, \tau) \, \int_0^\infty dz\, \frac{e^{-\nu z}}{z^{3/2}}. 
\end{align}
The results of our Laplace transform is apparent when the gain term transformed in \autoref{eqn: BDMPS_solution_gainterm_laplace}, is inserted back into the evolution equation of \autoref{eqn: BDMPS_solution_laplace_step1}
\begin{align}\label{eqn: BDMPS_solution_laplace_step2}
    \partial_t \tilde{F}(\nu,\tau) &= \tilde{F}(\nu, \tau) \, \int_0^\infty dz\, \frac{e^{-\nu z}}{z^{3/2}} - \tilde{F}(y,\tau) \, \int_0^\infty \frac{dz}{z^{3/2}} \nonumber \\
    &= \tilde{F}(\nu, \tau) \, \int_0^\infty dz\, \frac{(e^{-\nu z}-1)}{z^{3/2}} \nonumber \\
    &= \tilde{F}(\nu, \tau) \, (-2 \sqrt{\pi \nu}).
\end{align}
This is a simple differential equation. From energy conservation the initial condition is \(\tilde{F}_0 = 1\) - more precisely is the initial condition a delta function which takes into account partons ending with precisely zero momentum - but the solution becomes
\begin{equation}\label{eqn: BDMPS_solution_laplace_result}
    \tilde{F}(\nu,\tau) = e^{-2\sqrt{\pi \nu}\tau}.
\end{equation}
\textbf{Step five} - the final step of this calculation - is to do the inverse Laplace transformation on \autoref{eqn: BDMPS_solution_laplace_result}
\begin{align}
    F(y,\tau) &= \int_{c-i\infty}^{c+i\infty} \frac{d\nu}{2\pi i} e^{\nu y} \,\tilde{F}(\nu,\tau) \nonumber \\
    &= \frac{\tau}{y^{3/2}} \, \exp\left(-\pi \frac{\tau^2}{y}\right)
\end{align}
and reverting back \(F(y,\tau) = \sqrt{x}\, D(x,\tau)\) and \(y = 1-x\), our final solution is 
\begin{equation}\label{eqn: BDMPS_solution}
    D(x,\tau ) = \frac{\tau}{\sqrt{x}(1-x)^{3/2}}\, \exp\left(-\pi \frac{\tau^2}{1-x}\right).
\end{equation}
At this point, it is worth taking a deep breath and reflect on what just happened. We started out with the medium evolution equation and wrote it in terms of a new variable \(\xi\), this made it possible to solve the integral in the loss term. Another change of variables allowed us to perform a Laplace transform so that the gain and loss term obtained the same form, and the equation could therefore be solved as a differential equation in Laplace space. Finally, the inverse Laplace transform gave us the final expression in \autoref{eqn: BDMPS_solution}.

The solution given in \autoref{eqn: BDMPS_solution} is the full solution for the in-medium kinetic rate equation, using the reduced kernel. Contrasting the solution to the DGLAP equation of \autoref{eqn: DGLAP_solution_energyflowmedium}, which is valid only in the small \(x\) limit.

This general strategy was relatively similar to the solution of the DGLAP equation, as both aimed at writing the gain and loss terms in the same form by using some change of variable, and then performing a transformation in order to solve the equation, before transforming back to momentum space.

\subsection{Medium broadening of parton showers}
As mentioned at the start of the section we will now discuss how broadening can occur without inducing gluon radiation, and how to account for this in our evolution equations.

The evolution equations which we have presented for medium cascades so far, does not include any diffusion terms representing medium broadening happening between the individual splittings. This is represented by the partons in the distribution experiencing kicks form the medium, by the addition of a collision kernel \(\mathcal{C}(\vec l, \tau)\). If we were to include this broadening, the evolution equation for gluons could be written as~\cite{Probabilistic_picture, system_of_evolutionequations}
\begin{align}\label{eqn: BDMPSZ_3.19_ProbPic}
    \begin{split}
    \frac{\partial}{\partial \tau} D(x,\vec k, \tau) &= \int_{\vec l}\mathcal{C}(\vec l,\tau) \, D(x, \vec k - \vec l, \tau) \\
    &\quad + \int_x^1 dz\, \frac{1
    }{z^2} \sqrt{\frac{z}{x}} \mathcal{K}(z) D\left(\frac{x}{z}, \frac{\vec k}{z}, \tau\right) - \int_0^1 dz\, \frac{z}{\sqrt{x}} \mathcal{K}(z) D(x,\vec k, \tau)
    \end{split}
\end{align}
where \(\mathcal{C}(\vec l,t)\) is the elastic collision kernel, and \(\tau\) is defined as in \autoref{eqn: medium_tau_definiton}. By integrating this equation over transverse momentum \(\int_{\vec k} = \int \frac{d^2 \vec k}{(2\pi)^2}\), and setting \(D(x,t) = \int_{\vec k} D(x, \vec k, t)\), we are left with
\begin{align}
    \begin{split}
    \int_{\vec k} \frac{\partial}{\partial \tau} D(x,\vec k, \tau) &= \int_{\vec k} \int_{\vec l}\mathcal{C}(\vec l,\tau) \, D(x, \vec k - \vec l, \tau) \\
    &\quad +  \int_{\vec k} \int_x^1 dz\, \frac{1}{z^2} \sqrt{\frac{z}{x}} \mathcal{K}(z) D\left(\frac{x}{z}, \frac{\vec k}{z}, \tau\right) - \int_{\vec k} \int_0^1 dz\,\frac{z}{\sqrt{x}} \mathcal{K}(z) D(x,\vec k, \tau)
    \end{split}\\
    \begin{split}
    \frac{\partial}{\partial \tau} D(x, \tau) &= \int_{\vec l}\mathcal{C}(\vec l,\tau) \, D(x, \tau) \\
    &\quad + \int_x^1 dz\, \sqrt{\frac{z}{x}} \mathcal{K}(z) D\left(\frac{x}{z}, \tau\right) - \int_0^1 dz\, \frac{z}{\sqrt{x}} \mathcal{K}(z) D(x, \tau).
    \end{split}
\end{align}
Since \(\int_{\vec l} \mathcal{C}(\vec l, \tau) = 0\), this becomes simply, 
\begin{align}
     \frac{\partial}{\partial \tau} D(x, \tau) &= \int_x^1 dz\, \sqrt{\frac{z}{x}} \mathcal{K}(z) D\left(\frac{x}{z}, \tau\right) - \int_0^1 dz\, \frac{z}{\sqrt{x}} \mathcal{K}(z) D(x, \tau).
\end{align}
Which is equivalent to \autoref{eqn: BDMPS_solution_startingpoint}. We have therefore seen that when averaging over the final transverse momentum after the splitting process, the collision kernel can be disregarded. We are primarily concerned with the inclusive distribution integrated over transverse momentum, \(D(x,t) = \int_{\vec k} D(x, \vec k, t)\), and the broadening is therefore not relevant (in general) for our discussions.

\end{document}