\documentclass[main.tex]{subfiles}

\newcommand{\jet}{\text{jet}}
\newcommand{\nn}{\nonumber\\}

\begin{document}

\section{Introduction / Observables }
Blahblah...



\section{Parton Branching in Vacuum}

\subsection{Introduction / Observables}

\subsection{Splitting functions}\label{sec: derivation_splitting_functions_vacuum}
In this section we will introduce the splitting functions for our parton branchings, by identifying the QCD vertex factors of the correct Feynman diagrams, and then averaging over incoming and outgoing polarization to determine the splitting function. 

\begin{figure}[H]
    \centering
    \feynmandiagram [horizontal=a to b] {
    a [blob, label = \(\mathcal{M}\)] -- [gluon, edge label = a] b,
    f1 [particle = b] -- [gluon, edge label = \(\theta_b\)] b  -- [gluon, edge label= \(\theta_c\)] f2 [particle = c],
    f1 -- [opacity=0] f2,
    };
    \caption{ggg splitting}
    \label{fig: test3}
\end{figure}

Starting with the \(gg\) vertex of \autoref{fig: feynman_three-gluon_vertex}. When taking all the gluon momenta to be incoming, the matrix element will be given by \(\mathcal{M}_n\), the vertex factor, the two final gluon polarization vectors, and the gluon propagator for the initial gluon. 

\begin{align}
    \mathcal{M}_{n+1} &=  \left( \epsilon_b^\beta \epsilon_c^\gamma  \right) \left(gf_{\alpha\beta\gamma} V_{\alpha\beta\gamma}\right)  \, i D_F^{\mu\alpha}(t)\delta_{ij}  |\mathcal{M}_n|
\end{align}

where \(\alpha,\beta,\gamma\) are colour indices. We will now square this matrix element, and insert \(f^{\alpha\beta\gamma}f^{\alpha\beta\gamma}=C_A\). Finally we will rewrite the propagator in terms of two polarization vectors, and use the fact that it is proportional to \(1/t\), to obtain the following expression,

\begin{equation}\label{eqn: ggg_matrix_element}
    |\mathcal{M}_{n+1}|^2 \sim \frac{g^2}{t^2} C_A V_{\alpha\beta\gamma}^2  |\mathcal{M}_n|^2
\end{equation}

Continuing lets take a closer look at \(V_{\alpha\beta\gamma}\), which comes from the vertex factor. Since all momenta are incoming, we can use \(p_a +p_b +p_c = 0\), and the condition \(\epsilon_i \cdot p_i = 0\) to rewrite as, \footnote{Hello there.}

\begin{align}\label{eqn: ggg_vertex_factor_Ellis_5.6}
    V_{\alpha\beta\gamma} &= 
    (\epsilon_a^\alpha \epsilon_b^\beta g_{\alpha\beta})(\epsilon_c^\gamma (p_a-p_b)_\gamma) + 
    (\epsilon_b^\beta \epsilon_c^\gamma g_{\beta\gamma})(\epsilon_a^\alpha (p_b-p_c)_\alpha) + 
    (\epsilon_a^\alpha \epsilon_c^\gamma g_{\gamma\alpha})(\epsilon_b^\beta (p_c-p_a)_\beta) \nonumber \\
    &=
    (\epsilon_a \cdot \epsilon_b)(\epsilon_c^\gamma (-2p_b- p_c)_\gamma) + 
    (\epsilon_b \cdot \epsilon_c)(\epsilon_a^\alpha (2p_b -p_a)_\alpha) + 
    (\epsilon_a \cdot \epsilon_c)(\epsilon_b^\beta (2p_c-p_b)_\beta) \nonumber \\
    &= 
    (\epsilon_a \cdot \epsilon_b)(\epsilon_c \cdot -2p_b) + 
    (\epsilon_b \cdot \epsilon_c)(\epsilon_a \cdot 2p_b) +
    (\epsilon_a \cdot \epsilon_c)(\epsilon_b \cdot 2p_c) \nonumber\\
    &= -2\left(
    (\epsilon_a \cdot \epsilon_b)(\epsilon_c \cdot p_b) - 
    (\epsilon_b \cdot \epsilon_c)(\epsilon_a \cdot p_b) -
    (\epsilon_a \cdot \epsilon_c)(\epsilon_b \cdot p_c) 
    \right)
\end{align}

We will assume that the polarization vectors of the gluons are purely transverse, either as plane polarization states in the plane of branching \(\epsilon_i^{\text{in}}\), or normal to the plane of branching \(\epsilon_i^{\text{out}}\). We can therefore write down the following criteria for the gluon polarization, in relation to one another,

\begin{align}\label{eqn: gluon_polarization_criteria_ellis_5.7}
    \begin{split}
    \epsilon_i^{\text{in}} \cdot \epsilon_j^{\text{in}} &= \epsilon_i^{\text{out}} \cdot \epsilon_j^{\text{out}} = -1  \\
    \epsilon_i^{\text{in}} \cdot \epsilon_j^{\text{out}} &= \epsilon_i^{\text{out}} \cdot \, p_j \; = 0
    \end{split}
\end{align}

Now we need to relate the individual gluon polarization to the individual gluon momenta, as required by \autoref{eqn: ggg_vertex_factor_Ellis_5.6}. The incoming gluon moments are,

\begin{align}\label{eqn: ggg_incoming_gluon_momentum}
    \begin{split}
    p_a &= \left( E_a, E_a \, \sin \theta, 0, E_a \cos \theta \right)  \\
    p_b &= \left( E_b, E_b \, \sin \theta_b, 0, -E_b \cos \theta_b \right) \\
    p_c &= \left( E_c, -E_c \, \sin \theta_c, 0, E_c \cos \theta_c \right) 
    \end{split}
\end{align}

From the condition \(\epsilon_i \cdot p_i =0\) we can show that \(\epsilon_0 = \epsilon_i\), 
and the individual polarizations can be written as, 

\begin{align}\label{eqn: ggg_incoming_gluon_polarization}
    \begin{split}
    \epsilon^{\text{in}}_a &= \left( 0, 1, 0, 0\right)  \\
    \epsilon^{\text{in}}_b &= \left( 0, - \cos \theta_b, 0, \sin \theta_b \right) \\
    \epsilon^{\text{in}}_c &= \left( 0, \cos \theta_c, 0, \sin \theta_c \right) 
    \end{split}
\end{align}

Now we will combine the results in \autoref{eqn: ggg_incoming_gluon_momentum} and \autoref{eqn: ggg_incoming_gluon_polarization} to determine the relations between gluon momenta and polarization, in the small angle approximation where \(\sin \theta \approx \theta\) and \(\theta^2 \approx 0\), 

\begin{align}
    \begin{split}
    \epsilon^{\text{in}}_a \cdot p_b &= -E_b \sin \theta_b\\
        &= -E_b \theta_b \\
    \epsilon^{\text{in}}_b \cdot p_c &= E_c \sin \theta_c \cos \theta_b + E_c \sin \theta_b \cos \theta_c \\
    &= E_c \sin (\theta_b+\theta_c)  \\
    &= E_c \theta \\
    \epsilon^{\text{in}}_c \cdot p_b &= -E_b \sin \theta_b \cos \theta_c - E_b \sin \theta_c \cos \theta_b \\
    &= -E_b \sin(\theta_b+\theta_c) \\
    &= -E_b \theta
    \end{split}
\end{align}

Combining these results with \autoref{eqn: branched_energy_fractions_ellis_5.2} and \autoref{eqn: theta_momentum_conv_ellis_5.4}, we obtain our final relations between the polarization and momenta, 

\begin{align}\label{eqn: ggg_polarization_momta_relations_ellis_5.8}
    \epsilon^{\text{in}}_a \cdot p_b &= -z(1-z)\, E_a \theta \nonumber \\
    \epsilon^{\text{in}}_b \cdot p_c &= (1-z)\, E_a \theta\\
    \epsilon^{\text{in}}_c \cdot p_b &= -z\, E_a \theta \nonumber
\end{align}

Now we will use \autoref{eqn: gluon_polarization_criteria_ellis_5.7} and \autoref{eqn: ggg_polarization_momta_relations_ellis_5.8} to determine the vertex factor from \autoref{eqn: ggg_vertex_factor_Ellis_5.6} for different polarizations, 

\begin{align}
    V_{\epsilon_a^\text{in} \epsilon_b^\text{in} \epsilon_c^\text{in}} &= -2 \left[
    (\epsilon_a^\text{in} \cdot \epsilon_b^\text{in})(\epsilon_c^\text{in} \cdot p_b) - 
    (\epsilon_b^\text{in} \cdot \epsilon_c^\text{in})(\epsilon_a^\text{in} \cdot p_b) -
    (\epsilon_a^\text{in} \cdot \epsilon_c^\text{in})(\epsilon_b^\text{in} \cdot p_c) 
    \right] \nonumber\\
    &= -2 \left[
    (-1)(-z\,E_a\theta) - (-1)(-z(1-z)\,E_a\theta) - (-1)((1-z)\, E_a\theta) 
    \right] \nonumber \\
    &= -2 \left[ z - (z(1-z)) + (1-z) ) \right] \,E_a\theta \nonumber \\
    &= -2 \left[ (z-1)z+ 1 \right] \,E_a\theta
\end{align}

Since we are interested in the square of \(V_{\alpha\beta\gamma}\),

\begin{align}
    V_{\epsilon_a^\text{in} \epsilon_b^\text{in} \epsilon_c^\text{in}}^2 &= 4 \left[ (z-1)z+ 1
    \right]^2 \,\frac{t}{z(1-z)} \nonumber\\
    V_{\epsilon_a^\text{in} \epsilon_b^\text{in} \epsilon_c^\text{in}}^2 &= -4 t 
    \frac{((z-1)z+ 1)^2}{z(z-1)} \nonumber \\
    V_{\epsilon_a^\text{in} \epsilon_b^\text{in} \epsilon_c^\text{in}}^2 &= 4 t 
    \left( \frac{1-z}{z} + \frac{z}{(1-z)} + z(1-z) \right) \nonumber \\
    &= 4t \, F(z;\epsilon_a,\epsilon_b,\epsilon_c)
\end{align}

The final expression can now be inserted into \autoref{eqn: ggg_matrix_element} to obtain, 

\begin{equation}\label{eqn: ggg_matrix_element_ellis_5.9}
    |\mathcal{M}_{n+1}|^2 \sim \frac{4g^2}{t} C_A F(z;\epsilon_a,\epsilon_b,\epsilon_c) |\mathcal{M}_n|^2
\end{equation}

The function \(F(z;\epsilon_a,\epsilon_b,\epsilon_c)\) contains the information required for determining our splitting function, but it is depedent on the polarization of the three gluons. We therefore need to determine \(F(z;\epsilon_a,\epsilon_b,\epsilon_c)\) for all possible polarization, and then average over initial \(\epsilon_a\), and sum over final \(\epsilon_b, \epsilon_c\). The result is given from \cite{ellis_stirling_webber_1996}, and reproduced in \autoref{tab: ggg_polarization_dependence}.

\begin{table}[h]
    \centering
    \begin{tabular}[]{cccccc}
        \(\epsilon_a\) & \(\epsilon_b\) & \(\epsilon_c\)& \multicolumn{3}{c}{\(F(z;\epsilon_a,\epsilon_b,\epsilon_c)\)} \\
        \cmidrule(lr){1-3} \cmidrule(lr){4-6}
        in & in & in & \multicolumn{3}{c}{\(\frac{1-z}{z} + \frac{z}{(1-z)} + z(1-z)\)} \\[0.2cm]
        in & out & out & \multicolumn{3}{c}{\(z(1-z)\)} \\[0.2cm]
        out & in & out & \multicolumn{3}{c}{\(\frac{1-z}{z} \)} \\[0.2cm]
        out & out & in & \multicolumn{3}{c}{\(\frac{z}{(1-z)} \)} \\[0.2cm]
        \bottomrule
    \end{tabular}
    \caption{Polarization dependence of ggg-branching.}
    \label{tab: ggg_polarization_dependence}
\end{table}

The final part of determining the \(gg\) splitting function is to average over initial and sum over final sates using \autoref{tab: ggg_polarization_dependence}.

\begin{align}\label{eqn: ggg_polarizations_summed&averaged}
    \langle F \rangle &= \frac{(\epsilon_a^{\text{in}} \Sigma \epsilon_b\epsilon_c ) + (\epsilon_a^{\text{out}} \Sigma \epsilon_b\epsilon_c)}{2} \nonumber\\
    &= \frac{1}{2} \left( \{ \frac{1-z}{z} + \frac{z}{1-z} + z(1-z) \} + \{ z(1-z) \} \right) + \frac{1}{2} \left( \{\frac{1-z}{z} \} + \{ \frac{z}{1-z} \}\right) \} \nonumber\\
    &= \left( \frac{1-z}{z} + \frac{z}{1-z} + z(1-z) \right)
\end{align}

with \autoref{eqn: ggg_polarizations_summed&averaged} the only remaining part is to define our \(gg\) splitting function,

\begin{equation}\label{eqn: vacuum_ggg_splitting_function}
    \hat P_{gg}(z) = 2C_A \langle F \rangle = C_A \left[ \frac{1-z}{z} + \frac{z}{1-z} + z(1-z) \right]
\end{equation}

Where the factor \(2\) comes from the symmetry of the splitting, as the splitting function can give the value for either of the outgoing gluons.
There are some small enchantments for the matrix element for soft gluon emission in the plane of branching, but we will be using \autoref{eqn: vacuum_ggg_splitting_function} as our \(gg\) splitting function as it is good enough for our purposes.

The derivation for the \gqq and \qqg splitting functions are quite similar, but \textcolor{red}{we will leave them as an exercise for the reader.} They are given respectively by \autoref{eqn: vacuum_gqq_splitting_function} and \autoref{eqn: vacuum_qqg_splitting_function}.

\begin{equation}\label{eqn: vacuum_gqq_splitting_function}
    \hat P_{qg}(z) = n_f \, T_R \langle F \rangle = T_R \left[ z^2 + (1-z)^2 \right] 
\end{equation}

\begin{equation}\label{eqn: vacuum_qqg_splitting_function}
    \hat P_{qq} (z) = C_F \langle F \rangle = C_F \frac{1+z^2}{1-z}
\end{equation}

In addition to the colour factors, \(C_A = 3\), \(C_F=4/3\), and \(T_R=1/2\), we also need to account for the quark flavors in he \gqq vertex, assuming that we have five active quark flavors, we need to add a factor \(n_f = 5\) to this vertex.
\textcolor{red}{argument for the colour factors}

\subsection{Evolution variable}
We will now aim to introduce an evolution variable, such that the derivatives of the probability-density with respect to the variable, and momentum fraction, is equal the splitting function, 

\begin{equation}\label{eqn: evolutionvariable_condition}
    \frac{d \mathcal{P}(t,z)}{dt dz} = P(z)
\end{equation}

We begin by noting that the probability of branching in an interval changes by, 

\begin{align}
    d\mathcal{P}_{1\rightarrow 2} &= \frac{\alpha_s}{\pi} P(z) \, dz \, \frac{d\theta}{\theta} \nonumber \\
    &= \frac{\alpha_s}{2\pi} \frac{d\theta^2}{\theta^2} P(z) dz
\end{align}

here we have used, \(d\theta^2 =2\theta d\theta\), and we get the following equation, 

\begin{equation}\label{eqn: evolutionvariable_intermediate_condition1}
    \frac{d\mathcal{P}}{d\theta^2 dz} = \frac{\alpha_s}{2\pi} \frac{1}{\theta^2} P(z)
\end{equation}

Now we want to introduce an evolution variable which makes it possible to replace \(\theta^2\) in \autoref{eqn: evolutionvariable_intermediate_condition1}, such that the condition of \autoref{eqn: evolutionvariable_condition} is satisfied.
We threfore introduce the evolution variable of Dasgupta \cite{Dasgupta_2015} as \autoref{eqn: evolution_parameter_dasguptalike},

\begin{equation}\label{eqn: evolution_parameter_dasguptalike}
    t = \int _{\theta^2}^{\theta_\text{max}^2} \frac{d\theta'^2}{\theta'^2} \frac{\alpha_s(p_t \theta')}{2\pi} 
\end{equation}

and then using a fixed coupling constant \(\alpha_S \approx 0.1184\), we can write,

\begin{align} 
    t &= \frac{\alpha_s}{2\pi} \int_{\theta^2}^{R^2} \frac{d\theta'^2}{\theta'^2} \nonumber \\
    t&= \frac{\alpha_s}{2\pi} \left( ln R^2 \, - \, ln \theta^2\right) \nonumber \\
    \frac{dt}{d\theta^2} &= - \frac{\alpha_s}{2\pi} \frac{1}{\theta^2}
\end{align}

therefore, making the replacement in \autoref{eqn: evolutionvariable_intermediate_condition1} gives us,

\begin{align}
     \frac{d\mathcal{P}}{dt dz} &= \frac{d\mathcal{P}}{d\theta^2 dz} \frac{d\theta^2}{dt} \nonumber\\
    &= \left( \frac{\alpha_s}{2\pi} \frac{1}{\theta^2} P(z) \right) \left( -\frac{2\pi}{\alpha_s} \theta^2 \right) \nonumber \\
    &= -P(z)
\end{align}
\textcolor{red}{Minus sign?}

Which means we now have recovered \autoref{eqn: evolutionvariable_condition} which was the neccessary condition, our evolution variable is therefore given by \autoref{eqn: evolution_parameter_dasguptalike}. 

\newpage
\subsection{The DGLAP equation} %Ellis style
\subsubsection{Derivation of the DGLAP equation \textcolor{red}{ - Review!}}
We will now do a derivation of the DGLAP equation, closely following the method outlined in \cite{ellis_stirling_webber_1996}.

What we will do is examine the splitting process where quarks and gluons can enter and leave a given volume element by the splitting processes outlined in \autoref{tab: eneteringleaving}.

\begin{table}[h]
    \centering
    \begin{tabular}[]{cccc}
        \multicolumn{2}{c}{Quarks} & \multicolumn{2}{c}{Gluons} \\
        \cmidrule(lr){1-2} \cmidrule(lr){3-4}
        Entering & Leaving & Entering & leaving \\
        \cmidrule(lr){1-1}\cmidrule(lr){2-2}\cmidrule(lr){3-3}\cmidrule(lr){4-4}
        \(q \rightarrow qg\) & \(q \rightarrow qg\) & \(q \rightarrow qg\) & \(q \rightarrow qg\) \\
        \(g \rightarrow qq\) & \(\) & \(g \rightarrow qq\) & \(\) \\
        \bottomrule
    \end{tabular}
    \caption{Processes in which quarks and gluons can enter and leave a given volume element.}
    \label{tab: eneteringleaving}
\end{table}

%Now, we will look at the effects of multiple branchings, and the higher order contributions associated with these multiple small angle parton emissions. 
For now, we will restrict ourselves to one type of branching , such as the \(q\rightarrow qg\).

Using an evolution variable as given by \autoref{eqn: evolution_parameter_dasguptalike}.
The evolution equations can be determined by considering the pictorial representation (page 167) \cite{ellis_stirling_webber_1996}, where we observe the partons branching in and out of a volume \(\delta t \delta x\)

\begin{align}
    \delta f_{in} &= \frac{\delta t}{t} \int_x^1 dx' dz \hat{P}(z) f(x',t) \delta (x-zx') \nonumber\\
    &= \frac{\delta t}{t} \int_x^1 \frac{dz}{z} \hat{P}(z) f(x/z,t)
\end{align}

\begin{align}
    \delta f_{out} &= \frac{\delta t}{t} f(x,t) \int_0^x dx' dz \hat{P}(z) \delta (x'-zx) \nonumber \\
    &= \frac{\delta t}{t} f(x,t) \int_0^x dz\hat{P}(z) 
\end{align}

here, \(\delta(x-zx')\), is used to define the integration limits, for the higher momentum fractions, and \(\delta(x'-zx)\) is over the lower momentum fractions. 
Combining these two equations, we find the net change in the volume element, 

\begin{align} %Ellis 5.34
    \delta f(x,t) &= \delta f_{in} - \delta f_{out} \nonumber \\
    &= \left( \frac{\delta t}{t} \int_x^1 \frac{dz}{z} \hat{P}(z) f(x/z,t) \right) - \left( \frac{\delta t}{t} f(x,t) \int_0^x dz \hat{P}(z) \right)
\end{align}

which rewrites as, \textcolor{red}{WHERE DOES THIS t COME FROM? Ellis is only one who has it.}

\begin{equation}\label{eqn: DGLAP_ellisblaizot_with_limits}
    \frac{\partial}{\partial t} f(x,t) =\frac{1}{t} \int_x^1 \frac{dz}{z} \hat{P}(z) f(x/z,t) - \frac{1}{t} \int_0^x dz \hat{P}(z) f(x,t) 
\end{equation}
    

Since integrands vanish for \(z<x\) and \(x<z\) respectively it can also be written as, 

\begin{equation}\label{eqn: DGLAP_ellisblaizot}
    \frac{\partial}{\partial t} f(x,t) = \frac{1}{t} \int_0^1 dz \hat{P}(z) \, \left( \frac{1}{z} f(x/z,t) - f(x,t) \right)
\end{equation}

which is the DGLAP equation as given by equation (5.34) of Ellis \cite{ellis_stirling_webber_1996}. This form of the DGLAP equation is convenient for numerical methods. For working analytically it is however preferred to introduce the plus prescription, defined as,
\begin{equation}\label{eqn: ellis_plusprescription}
    \int_0^1 dx \frac{f(x)}{(1-x)_+} = \int_0^1 \frac{f(x)-f(1)}{1-x}
\end{equation}

we can rewrite \autoref{eqn: DGLAP_ellisblaizot} for the parton density with the now regularized splitting function \(P(z) = \hat P(z)_+\). \textcolor{red}{Remove the 'Ellis t'...}

\begin{equation}\label{eqn: dglap_ellis_style}
    t \frac{\partial }{\partial t} f(x,t) = \int_0^1 \frac{dz}{z} \, P(z) \, f(x/z, t)
\end{equation}

Which is the DGLAP equation as given by equation (5.36) of Ellis \cite{ellis_stirling_webber_1996}.

It has the following quantities, 

\begin{itemize}
    \item \(f(x,t)\) - Distribution of parton momenta in the incoming hadron A, probed at scale t. (For time-like branching it can represent outgoing parton momenta).
    \item \(P(z)\) - Regularized parton splitting function.
\end{itemize}


\subsubsection{Rewriting the DGLAP equation in terms of the Sudakov form factor}
If we consider \autoref{eqn: DGLAP_ellisblaizot}, which is the DGLAP equation in terms of the unregularized splitting functions, %Ellis 5.34

\begin{align}\tag{\ref{eqn: DGLAP_ellisblaizot}}
    \frac{\partial}{\partial t} f(x,t) &= \int_{\epsilon}^{1-\epsilon} dz \, \hat{P}(z) \left( \frac{1}{z} f(x/z ,t) - f(x,t) \right) \nonumber \\
    &= \int_{\epsilon}^{1-\epsilon} \frac{dz}{z} \, \hat{P}(z) f(x/z ,t) - \int_{\epsilon}^{1-\epsilon} dz \, \hat{P}(z) f(x,t)
\end{align}
\textcolor{red}{Remove the 'Ellis t'...}

and introduce the Sudakov form factor as in \cite{Dasgupta_2015},

\begin{align}\label{eqn: sudakov_form_factor_dasguptalike}
    \Delta (t) &= \text{exp}\left(-t\, \int_{\epsilon}^{1-\epsilon}dz \, \hat P(z)\right) 
\end{align}

with the derivative,
\begin{align}
    \frac{\partial}{\partial t} \Delta (t) &= (- \int_{\epsilon}^{1-\epsilon} dz\, \hat P(z) )\cdot  \Delta(t) \nonumber\\
    \frac{\partial}{\partial t} \frac{1}{\Delta(t)}&= \int_{\epsilon}^{1-\epsilon} dz\, \hat P(z) \cdot  \frac{1}{\Delta(t)}
\end{align}

we can then write \autoref{eqn: DGLAP_ellisblaizot} \textcolor{red}{but without the 'Ellis t'...} as, 

\begin{align}\label{eqn: DGLAP_evolutioneq_unregularized_differential_ellis}
    \frac{\partial}{\partial t} f(x,t) + \int_{\epsilon}^{1-\epsilon} dz \, \hat{P}(z) f(x,t) &= \int_{\epsilon}^{1-\epsilon} \frac{dz}{z} \, \hat{P}(z) f(x/z ,t) \nonumber \\
    \frac{\partial}{\partial t} f(x,t) + \Delta(t) f(x,t) \frac{\partial}{\partial t} \frac{1}{\Delta(t)} &= \int_{\epsilon}^{1-\epsilon} \frac{dz}{z} \, \hat{P}(z) f(x/z ,t) \nonumber \\
    \frac{1}{\Delta(t)} \frac{\partial}{\partial t} f(x,t) - f(x,t) \frac{\partial}{\partial t} \frac{1}{\Delta(t)} &= \frac{1}{\Delta(t)} \int_{\epsilon}^{1-\epsilon} \frac{dz}{z} \, \hat{P}(z) f(x/z ,t) \nonumber \\
    \frac{\partial }{\partial t} \frac{f(x,t)}{\Delta (t)} &= \frac{1}{\Delta(t)} \int_{\epsilon}^{1-\epsilon} \frac{dz}{z} \, \hat P(z) \, f(x/z, t)
\end{align}


\autoref{eqn: DGLAP_evolutioneq_unregularized_differential_ellis} corresponds to equation (5.48) of \cite{ellis_stirling_webber_1996} \textcolor{red}{but without the 'Ellis t'...}. Continuing we can rewrite this from a differential form to an integral form,

\begin{align}
    \int_{t_0}^t dt' \frac{\partial}{\partial t} \frac{f(x,t')}{\Delta(t')} = &\int_{t_0}^{t} \frac{dt'}{t'} \frac{1}{\Delta(t')} \int_0^1 dz \frac{\alpha_S}{2\pi} \, \hat{P}(z) \frac{1}{z} f(x/z, t) \nonumber\\
    \frac{f(x,t)}{\Delta(t)} - \frac{f(x,t_0)}{\Delta(t_0)} = &\int_{t_0}^{t} \frac{dt'}{t'} \frac{1}{\Delta(t')} \int_0^1 dz \frac{\alpha_S}{2\pi} \, \hat{P}(z) \frac{1}{z} f(x/z, t) \nonumber\\
    \frac{f(x,t)}{\Delta(t)} - f(x,t_0) &= \int_{t_0}^{t} \frac{dt'}{t'} \frac{1}{\Delta(t')} \int_0^1 dz \frac{\alpha_S}{2\pi} \, \hat{P}(z) \frac{1}{z} f(x/z, t)
\end{align}

And we end up with our desired evolution equation on integral form, 

\begin{equation}\label{eqn: DGLAP_evolutioneq_unregularized_integral_ellis}
    f(x,t) =  \Delta(t) f(x,t_0) + \int_{t_0}^{t} \frac{dt'}{t'} \frac{\Delta(t)}{\Delta(t')} \int \frac{dz}{z} \frac{\alpha_S}{2\pi} \, \hat{P}(z) \, f(x/z, t')
\end{equation}



\subsubsection{Solution of DGLAP cascade (Blaizot app B) \textcolor{red}{Do this}}
We will now solve equation (2.10) in Mellin space.

\begin{equation}\label{eqn: blaizot_2.10}
    \frac{\partial}{\partial t} D(x,t) = \bar a \int_z^1 \frac{dz}{z(1-z)} D(x/z,t) - \bar a \int_0^1 \frac{dz}{(1-z)} D(x,t)
\end{equation}

the mellin transforms are defined as, \cite{Energy_flow_medium_cascade_2016},
\begin{equation*}\label{eqn: mellin_transforms}
    \Tilde{D}(\nu, t) = \int_0^1 dx\, x^{\nu-1} \, D(x,t) \qquad \text{, and }\quad D(x,t) = \int_{c-i\infty}^{c+i\infty} \frac{d\nu}{2\pi i}\, x^{-\nu} \, \Tilde{D}(\nu, t)
\end{equation*}

\begin{align*}
    \frac{\partial}{\partial t} \int_{c-i\infty}^{c+i\infty} \frac{d\nu}{2\pi i}\, x^{-\nu} \, \Tilde{D}(\nu, t) &= \bar a \int_z^1 \frac{dz}{z(1-z)} \int_{c-i\infty}^{c+i\infty} \frac{d\nu}{2\pi i}\, (x/z)^{-\nu} \, \Tilde{D}(\nu, t) \\
    &- \bar a \int_0^1 \frac{dz}{(1-z)} \int_{c-i\infty}^{c+i\infty} \frac{d\nu}{2\pi i}\, x^{-\nu} \, \Tilde{D}(\nu, t)
\end{align*}

\begin{equation*}
    \frac{\partial}{\partial t} \int_0^1 dx\, x^{\nu-1} \, D(x,t) = -\int_0^1 \frac{1}{1-z}\, dz +\int_0^1 \frac{z^\nu}{1-z}\, dz 
\end{equation*}


\begin{equation*}
    \frac{\partial}{\partial t} \Tilde{D}(\nu, t) = -\int_0^1 \frac{1}{1-z}\, dz +\int_0^1 \frac{z^\nu}{1-z}\, dz 
\end{equation*}

\begin{equation*}
    \frac{\partial}{\partial t} \Tilde{D}(\nu, t) = -\int_0^1 \frac{1-z^\nu}{1-z}\, dz 
\end{equation*}

\begin{equation*}
    \frac{\partial}{\partial t} \Tilde{D}(\nu, t) = -(\psi(\nu)+\gamma) = \frac{\Gamma'(\nu)}{\Gamma(\nu)}
\end{equation*}

\textcolor{red}{WRONG!}


\subsection{DGLAP Konrad }

The evolution equations that I work with are the following. I think they are consistent with the literature (although some places there are typos etc.).

The functions $p|_{a\to b(z)+c(1-z)} \equiv p_{ba}(z)$ are the un-regularized Altarelli-Parisi splitting functions, listed here
\begin{align}
p_{qq}(z) &= C_F \frac{1+z^2}{1-z} \,,\\
p_{gq}(z) &= C_F \frac{1+(1-z)^2}{z} \,,\\
p_{gg}(z) &= C_A \left[\frac{z}{1-z} + \frac{1-z}{z} + z(1-z) \right]\,, \\
p_{qg}(z) &=  n_f T_R \big[ z^2 + (1-z)^2 \big] \,,
\end{align}
with $C_A = N_c =3$, $C_F = 4/3$ and $T_R =1/2$, and $n_f$ is the number of active quark flavors.

The inclusive jet distributions for quarks and gluons follows from a set of coupled Dokshitzer-Gribov-Lipatov-Altarelli-Parisi (DGLAP) evolution equations \cite{Dasgupta_2015}. Consider the density of energy carried by particles inside a quark or gluon ($i=q,g$) jet with total energy (transverse momentum in LAB frame) $E=p_T$ and a given cone-size $R$, i.e.
\begin{equation}
    D_{i}(z,t) = z\frac{\dd N}{\dd z \, \dd t}\,,
\end{equation}
carrying the momentum fraction $z = \omega/p_T$ and emitted at an angle $\theta<R$, so that $t = \ln (\theta/R)$.
Explicitly, the quark jet evolution equation reads,
\begin{align}
\label{eq:qevol-1}
\frac{\partial}{\partial t} D_{q} (z,t) & = \int_0^1 \dd z'\, \frac{\alpha_s}{\pi}p_{qq}(z')\left[ D_{q}\left( \frac{z}{z'}, t \right) - D_{q}(z,t) \right] + \int_0^1 \dd z'\, \frac{\alpha_s}{\pi}p_{gq}(z') D_{g} \left( \frac{z}{z'} ,t \right) \,,
\end{align}
where \(\alpha_s = \alpha_s(k_t=z(1-z)\theta p_T)\) . Next, we turn to the distribution of particles inside gluon jets, whose evolution equation reads,
\begin{align}
\label{eq:gevol-1}
\frac{\partial}{\partial t} D_{g}(z,t) &= \int_0^1 \dd z' \, \frac{\alpha_s}{\pi} p_{gg}(z') \left[2 D_{g} \left(\frac{z}{z'} ,t \right) - D_{g}(z,t) \right] + \int_0^1 \dd z'\,\frac{\alpha_s}{\pi} p_{qg}(z') \left[ D_{q} \left(\frac{z}{z'},t \right) - D_{g}(z,t) \right] \,,
\end{align}
where the factor \(2\) next to the gluon distribution in the first line arises due to symmetry. 

Using the well-known manipulations, we can immediately identify the relevant Sudakov factors, as
\begin{align}
    \Delta_q(t) &= \exp \left\{ - \int_0^t \dd t'\int_\epsilon^{1-\epsilon} \dd z\, \frac{\alpha_s}{\pi} \,p_{qq}(z) \right\} \,,\\
    \Delta_g(t) &= \exp \left\{ - \int_0^t \dd t'\int_\epsilon^{1-\epsilon} \dd z\, \frac{\alpha_s}{\pi} \,\left[ p_{gg}(z) + p_{qg}(z) \right] \right\} \,.
\end{align}
That agrees with the formulas for the Sudakovs above.

\subsubsection{Analytical solution}

In order to find an analytical solution, we can introduce a set of simplifications. 

Here, we quote the final result. For a pure gluon emissions (no $g\to q \bar q$ splitting), we find
\begin{equation}
    D_i(x|p_T,R) = \frac{C_i}{C_A}\sqrt{\frac{\bar\alpha (Y - \ell)}{\ell}} I_1 \left( 2 \sqrt{\bar \alpha \,\ell (Y- \ell)} \right) \,,
\end{equation}
where \(\bar \alpha =\alpha_s C_A/\pi\), \(\mathcal{l} = \ln 1/x\) and \(Y= \ln[ (p_T R)^2/Q_0^2]\). Here, again \(p_T\) and \(R\)
 are the jet energy and cone-size, while \(Q_0 \sim 0.2-1\) GeV represents an IR transverse momentum scale.


\newpage
\section{Parton Branching in Medium}
Now that we have encountered parton showers in vacuum, and will move on to discuss cascades that develop in a dense QCD matter. \cite{Energy_flow_medium_cascade_2016} The new medium showers will be described using the BDMPS equation, and a suitable shower program will later be developed.
In this section we will work with an Ideal BDMPS, where we allow gluons to split all the way down to zero energy.

\subsection{Differences between DGLAP and BDMPS cascades}
Before introducing the BDMPS equation, a discussion is required for covering the main differences between medium and vacuum showers. In the vacuum discussion the first point of interest was the evolution variable which was a dimensionless quantity used to conveniently write the evolution equations. For the medium showers this will be replaced by something called the characteristic time \(t_*\), which has dimension equal to the actual time \([GeV^{-1}]\), and defined in \autoref{eqn: characteristic_time}. The characteristic time is the time it takes for a gluon of energy \(\omega\) to radiate most of its energy into soft gluons. It is also called the stopping time.

\begin{equation}\label{eqn: characteristic_time}
    t_* \equiv \frac{1}{\bar \alpha} t_{br}(E) = \frac{\pi}{\alpha_S N_C} \sqrt{\frac{E}{\hat q}}
\end{equation}

Here \(t_{br}(E)\) is the typical time which is the time it takes a gluon of energy E to branch into two gluons, and \(\bar \alpha \equiv \alpha_S N_C / \pi\). The energy is denoted \(E\), and the jet-quenching parameter \(\hat q\), as usual.

An important feature of the medium cascades is that the branching rate increases along the cascade, in contrast to the DGLAP evolution of the vacuum showers, where the branching rate is constant along the cascade. This is apparent when looking at the characteristic time, as the energy \(\omega\) of a given gluon decreases, the expected time for a branching to occur also decreases. This means that the branchings are accelerating, and it takes a finite time to transport a finite amount of energy from the leading particle to soft gluons. Energy is therefore effectively \textcolor{red}{transported towards large angles} which contrasts the strong angular ordering of QCD cascades in vacuum.

Another consequence of the characteristic time is a scaling behavior of the form \(D(\omega) \sim 1/\sqrt{\omega} \). This scaling relates to the existence of a stationary solution of the energy distribution. 

%\begin{equation}
%    D_{st} (\omega) = \frac{t_*(\omega)}{\omega} \qquad , \quad \mathcal{F}(\omega) = \frac{\partial \epsilon(\omega)}{\partial t} \sim \frac{\omega D_{st}(\omega)}{t_*(\omega)} =\text{const.}
%\end{equation}

The splitting functions are also slightly different from the vacuum counterparts, taking into account that \(C_A=N_C\), the medium \(gg\) splitting function \(\mathcal{K}_{gg}(z)\) will take the form, 

\begin{align}\label{eqn: vacuumtomedium_ggg_splitting_relation}
    \frac{\alpha_S}{\pi} \mathcal{K}_{gg}(z) &= \frac{\alpha_S}{2 \pi} 2 P_{gg}(z) \sqrt{N_C} \sqrt{\frac{1-z(1-z)}{z(1-z)}} \nonumber \\
    &= \bar a \sqrt{N_C} \frac{\left[1-z(1-z)\right]^2}{z(1-z)}  \sqrt{\frac{1-z(1-z)}{z(1-z)}} \nonumber \\
    &= \bar a \sqrt{N_C} \, \mathcal{K}(z)
\end{align}

\(\bar \alpha\) is defined as before, and the splitting function has been written in terms of the full \(gg\) splitting kernel \(\mathcal{K}(z)\), which should not be confused with the splitting function \(\mathcal{K}_{gg}(z)\). However it is generally sufficient - and more convenient - to work with a reduced kernel given her in \autoref{eqn: ggg_medium_reduced_kernel}, 

\begin{equation}\label{eqn: ggg_medium_reduced_kernel}
    \mathcal{K}(z) = \frac{1}{\left( z(1-z)\right)^{3/2}}
\end{equation}


\subsection{The BDMPS equation}\label{sec: BDMPS_theory}
The necessary ingredients for constructing the BDMPS equation has now been introduced. It is expected that the form of the evolution equation in medium, will be very similar to the vacuum version. Notable differences from the DGLAP is that the BDMPS will be expressed in terms of the inclusive gluon distribution integrated over transverse momentum \(D(x,t) = x f(x,t)\), and the characteristic time \(t_*\). For a full derivation of see \cite{Probabilistic_picture_for_medium-induced_jet_evolution}, \textcolor{red}{Do my own derivation?}, here we simply quote the results. 

\begin{equation}\label{eqn: BDMPS_2.5_Blaizot}
    \frac{\partial}{\partial t} D(x,t) = \frac{1}{t_*} \int_x^1 dz\, \mathcal{K}(z)\, \sqrt{\frac{z}{x}}\, D\left(\frac{x}{z}, t\right) -\frac{1}{t_*} \int_0^1 dz\, \mathcal{K}(z)\, \frac{z}{\sqrt{x}}\, D\left(x,t\right)
\end{equation}

Lets call this the full version of the BDMPS equation as most of the quantities we know are present in this form. 
There are however other forms of \autoref{eqn: BDMPS_2.5_Blaizot} which will be more convenient. One of them is obtained by absorbing the factor \(\sqrt{x}\) into the characteristic time, such that \(t_*(x) = t_* \sqrt{x}\). This gives an effective time scale for the branching of a gluon carrying a fraction \(x\) of the initial energy, which is one of the properties of the BDMPS cascade.

\begin{equation}\label{eqn: BDMPS_2.8_Blaizot}
    \frac{\partial}{\partial t} D(x,t) = \int_x^1 dz\, \mathcal{K}(z)\, \frac{D\left(\frac{x}{z}, t\right)}{t_*(\frac{x}{z})} - \frac{D\left(x,t\right)}{t_*(x)} \int_0^1 dz\, z\, \mathcal{K}(z)
\end{equation}

Another simplification can be obtained by noting that the kernel is independent of time. The evolution equation can therefore be simplified by introducing a new dimensionless variable \(\tau\) which accounts for the energy dependence and time scale of the branchings, 

\begin{equation}\label{eqn: medium_tau_definiton}
    \tau = \frac{t}{t_*}= \bar a \sqrt{\frac{\hat{q}}{E}} t
\end{equation}

using this new variable \autoref{eqn: BDMPS_2.5_Blaizot} can be written as, 

\begin{equation}\label{eqn: BDMPS_solution_startingpoint}
    \frac{\partial}{\partial \tau} D(x, \tau) = \int_x^1 dz \,\mathcal{K}(z) \sqrt{\frac{z}{x}} D(\frac{x}{z}, \tau) - \int_0^1 dz \,\mathcal{K}(z) \frac{z}{\sqrt{x}} D(x,t)
\end{equation}



\subsubsection{Derivation of the BDMPS equation}
Attempt to recreate the results of 1311.5823, \cite{Probabilistic_picture_for_medium-induced_jet_evolution}.

\subsubsection{Rewriting the BDMPS equation using a Sudakov-like form-factor}\label{sec: medium_sudakov}
We will now start with \autoref{eqn: BDMPS_2.8_Blaizot}, and rewrite it in terms of a sudakov form-factor, like we did for the DGLAP equation. 


If we introduce a Sudakov form-factor, \textcolor{red}{Possible integral for t...}

\begin{align}\label{eqn: BDMPS_sudakov}
    \Delta (t) &= \exp \left( -\frac{t}{t_*(x)} \int_0^1 \, dz\, z \mathcal{K}(z) \right)
\end{align}

with the derivatives, 

\begin{align}
    \frac{\partial}{\partial t} \Delta (t) &= - \frac{1}{t_*(x)} \int_0^1 \, dz\, z \mathcal{K}(z) \cdot \Delta(t) \nonumber\\
    \frac{\partial}{\partial t} \frac{1}{\Delta (t)} &= \frac{1}{t_*(x)} \int_0^1 \, dz\, z \mathcal{K}(z) \cdot \frac{1}{\Delta(t) }
\end{align}

Starting by rewriting \autoref{eqn: BDMPS_2.8_Blaizot},

\begin{align}
    \frac{\partial}{\partial t} D(x,t) + \frac{D\left(x,t\right)}{t_*(x)} \int_0^1 dz\, z\, \mathcal{K}(z) &= \int_x^1 dz\, \mathcal{K}(z)\, \frac{D\left(\frac{x}{z}, t\right)}{t_*(x/z)} \nonumber\\
    \frac{\partial}{\partial t} D(x,t) + D(x,t) \frac{\partial}{\partial t} \frac{1}{\Delta(t)} &= \int_x^1 dz\, \mathcal{K}(z)\, \frac{D\left(\frac{x}{z}, t\right)}{t_*(x/z)} \nonumber\\
    \Delta(t) \frac{\partial}{\partial t} \left( \frac{D(x,t)}{\Delta(t)} \right) &= \int_x^1 dz\, \mathcal{K}(z)\, \frac{D\left(\frac{x}{z}, t\right)}{t_*(x/z)} \nonumber\\
    \frac{\partial}{\partial t} \left( \frac{D(x,t)}{\Delta(t)} \right) &= \frac{1}{\Delta(t)} \, \int_x^1 dz\, \mathcal{K}(z)\, \frac{D\left(\frac{x}{z}, t\right)}{t_*(x/z)} 
\end{align}

integrating out the t integral, 

\begin{align}
    \frac{D(x,t)}{\Delta(t)} - \frac{D(x,t_0)}{\Delta(t_0)} &= \int_{t_0}^t \frac{dt'}{\Delta(t')} \, \int_x^1 dz\, \mathcal{K}(z)\, \frac{D\left(\frac{x}{z}, t'\right)}{t_*(x/z)} \nonumber\\
    D(x,t) &= D(x,t_0)\, \frac{\Delta(t)}{\Delta(t_0)} + \int_{t_0}^t dt' \, \frac{\Delta(t)}{\Delta(t')} \, \int_x^1 dz\, \mathcal{K}(z)\, \frac{D\left(\frac{x}{z}, t'\right)}{t_*(x/z)} 
\end{align}

if we consider the initial time \(t_0 = 0\), then \(\Delta(t_0) = 1\), and we get an equation which is the medium equivalent of \autoref{eqn: DGLAP_evolutioneq_unregularized_integral_ellis}, 

\begin{equation}
    D(x,t) = D(x,t_0)\, \Delta(t) + \int_{t_0}^t dt' \, \frac{\Delta(t)}{\Delta(t')} \, \int_x^1 dz\, \mathcal{K}(z)\, \frac{D\left(\frac{x}{z}, t'\right)}{t_*(x/z)} 
\end{equation}


\subsubsection{Solution of BDMPS cascade }
We will now solve the medium evolution equation by closely following the method outlined in \cite{Energy_flow_medium_cascade_2016}.
The starting point for solving the medium evolution equation is \autoref{eqn: BDMPS_solution_startingpoint}, where \(\mathcal{K}(z)\) is the reduced kernel given in \autoref{eqn: ggg_medium_reduced_kernel}, and \(\tau \) is defined as in \autoref{eqn: medium_tau_definiton}.

The first step is to perform a change of variable such that \(\xi = \frac{x}{z}\) in the gain term and \(\xi = xz\) in the loss term. 

\begin{align}\label{eqn: BDMPS_solution_gainterm_changeofvariable}
    \mathbb{G} &= \int_x^1 dz \mathcal{K}(z) \sqrt{\frac{z}{x}} D(\frac{x}{z}, \tau) \qquad,\quad \xi = \frac{x}{z} \nonumber \\
    &= \int_1^x d\xi (-\frac{x}{\xi^2}) \mathcal{K}(\frac{x}{\xi}) \sqrt{\frac{1}{\xi}} D(\xi, \tau) \nonumber \\
    &= \int_x^1 d\xi \,\frac{x}{\xi^(5/2)} \mathcal{K}(\frac{x}{\xi}) D(\xi, \tau)
\end{align}


\begin{align}\label{eqn: BDMPS_solution_lossterm_changeofvariable}
    \mathbb{L} &= - \int_0^1 dz \mathcal{K}(z) \frac{z}{\sqrt{x}} D(x,\tau)\qquad , \quad \xi = xz \nonumber \\
    &= - \int_0^x d\xi (\frac{1}{x}) \mathcal{K}(\frac{\xi}{x}) \frac{\xi}{x^(3/2)} D(x,\tau) \nonumber \\
    &= - \int_0^x d\xi \,\frac{\xi}{x^(5/2)} \mathcal{K}(\frac{\xi}{x}) D(x,\tau)
\end{align}

in these equations a common splitting function can be identified as, 

\begin{align}\label{eqn: BDMPS_solution_splittingfunction_xivariable}
    P(x,\xi) &= \frac{x}{\xi^(5/2)} \mathcal{K}(\frac{x}{\xi}) \nonumber \\
    %&= \frac{x}{\xi^(5/2)} \frac{1}{\left[\frac{x}{\xi}(1-\frac{x}{\xi})\right]^{(3/2)}} \nonumber \\
    %&= \frac{x}{\xi^(5/2)} \frac{1}{ \frac{1}{\xi^3} \left[x(\xi-x)\right]^{(3/2)}} \nonumber \\
    &= \sqrt{\frac{\xi}{x}} \frac{1}{(\xi-x)^{(3/2)}}
\end{align}

and \autoref{eqn: BDMPS_solution_startingpoint} can therefore be written as, 

\begin{equation}\label{eqn: BDMPS_solution_evoleqn_with_xisplitfunc}
    \partial_\tau D(x,\tau) = \int_x^1 d\xi \,P(x,\xi) D(\xi, \tau) - \int_0^x d\xi \,P(\xi,x) D(x,\tau)
\end{equation}

note that \(P(x,\xi) \neq P(\xi, x)\). 
Now that the gain and loss terms are written in a convenient and symmetrical way, the next step is deal with the integral of the loss term.

\begin{equation}
    \int_0^x d\xi \, \sqrt{\frac{1}{\xi}} \frac{1}{(x-\xi+\epsilon)^{(3/2)}} = \frac{1}{\sqrt{\epsilon}} \frac{2x}{x+\epsilon} \approx \frac{2}{\sqrt \epsilon} - \frac{2\sqrt{\epsilon}}{x} + \mathcal{O}(\epsilon^{3/2}) 
\end{equation}

In the limit \(\epsilon \rightarrow 0\), the first term is divergent, and all subleading terms vanish for any finite value of \(x\). Therefore, the sole purpose of the loss term (in these variables) is to remove the singularity of the gain term.  We can therefore replace the integral in the loss term with the following, 

\begin{equation}
    \mathbb{L} = - D(x,\tau) \, \int_0^\infty \frac{dz}{z^{3/2}}
\end{equation}

The third step is to introduce another change of variables \(y = 1-x\), and a re-scaling of the distribution \(F(y, \tau ) = \sqrt{x} D(x,\tau)\). Starting by multiplying everything by \(\sqrt{x}\) and inserting the new loss term, \autoref{eqn: BDMPS_solution_evoleqn_with_xisplitfunc} becomes,

\begin{align}\label{eqn: BDMPS_solution_evoleqn_with_F(y,tau)}
    \partial_\tau \sqrt{x} D(x,\tau) &= \int_x^1 d\xi \,\sqrt{x}\, \sqrt{\frac{\xi}{x}} \frac{1}{(\xi-x)^{(3/2)}} \,D(\xi, \tau) - \sqrt{(1-y)}\, D((1-y),\tau) \, \int_0^\infty \frac{dz}{z^{3/2}} \nonumber \\
    \partial_\tau F(y,\tau) &= \int_x^1 d\xi \, \frac{1}{(\xi-(1-y))^{(3/2)}} \sqrt{\xi} \,D(\xi, \tau) - \sqrt{(1-y)}\, D((1-y),\tau) \, \int_0^\infty \frac{dz}{z^{3/2}}
\end{align}

making making the replacement \(\tilde{\xi}  = 1- \xi\), and utilizing the property \(F(y, \tau ) = \sqrt{1-y} \,D(1-y,\tau)\),

\begin{align}\label{eqn: BDMPS_solution_evoleqn_Laplace_ready}
    \partial_\tau F(y,\tau) &= \int_0^{y} d\tilde{\xi} \, \frac{1}{(y-\tilde{\xi})^{(3/2)}} \sqrt{1-\tilde{\xi}} \,D(1-\tilde{\xi}, \tau) - F(y,\tau) \, \int_0^\infty \frac{dz}{z^{3/2}} \nonumber \\
    &= \int_0^{y} d\tilde{\xi} \, \frac{1}{(y-\tilde{\xi})^{(3/2)}} F(\tilde{\xi}, \tau) - F(y,\tau) \, \int_0^\infty \frac{dz}{z^{3/2}}
\end{align}

Step 4 is to extend the limits of the domain for \(F(y,\tau)\) from \(y\in[0,1] \rightarrow y \in [0, \infty]\), and Laplace transform our evolution equation. Defining the Laplace transform as, 

\begin{equation}\label{eqn: BDMPS_solution_Laplace_definition}
    \tilde{F}(\nu, \tau) = \int_0^\infty dy \, e^{-\nu y}\, F(y,\tau)
\end{equation}

performing the Laplace transform on \autoref{eqn: BDMPS_solution_evoleqn_Laplace_ready},

\begin{align}\label{eqn: BDMPS_solution_laplace_step1}
    \partial_t \tilde{F}(y,\tau) &= \int_0^\infty dy\, e^{-\nu y}\,  \int_0^{y} d\tilde{\xi} \, \frac{1}{(y-\tilde{\xi})^{(3/2)}} F(\tilde{\xi}, \tau) - \int_0^\infty dy \, e^{-\nu y}\, F(y,\tau) \, \int_0^\infty \frac{dz}{z^{3/2}} \nonumber \\
    &= \int_0^\infty dy\, \int_0^{y} d\tilde{\xi} \, e^{-\nu y} \frac{1}{(y-\tilde{\xi})^{(3/2)}} F(\tilde{\xi}, \tau) - \tilde{F}(y,\tau) \, \int_0^\infty \frac{dz}{z^{3/2}}
\end{align}

Since the loss term only had one \(y\) dependence, the Laplace transform went very smoothly. When dealing with the gain term it is nessecary to make some changes to the integration boundaries \(\int_0^\infty dy\, \int_0^y d\xi \,\rightarrow\, \int_\xi^\infty dy\, \int_0^\infty d\xi\), and then introduce another change of variable \(z = y-\tilde{\xi}\),

\begin{align}\label{eqn: BDMPS_solution_gainterm_laplace}
    \mathbb{G} &= \int_0^{\infty} d\tilde{\xi} \,F(\tilde{\xi}, \tau) \int_\xi^\infty dy\,  e^{-\nu y} \frac{1}{(y-\tilde{\xi})^{(3/2)}} \nonumber \\
    &= \int_0^{\infty} d\tilde{\xi} \,F(\tilde{\xi}, \tau) \int_0^\infty dz\, \frac{e^{-\nu (z+\tilde{\xi})}}{z^{(3/2)}} \nonumber \\
    &= \int_0^{\infty} d\tilde{\xi} e^{-\nu \tilde{\xi}} \,F(\tilde{\xi}, \tau) \int_0^\infty dz\, \frac{e^{-\nu z}}{z^{(3/2)}} \nonumber \\
    &= \tilde{F}(\nu, \tau) \, \int_0^\infty dz\, \frac{e^{-\nu z}}{z^{(3/2)}} 
\end{align}

The results of our Laplace transform is apparent when the gain term transformed in \autoref{eqn: BDMPS_solution_gainterm_laplace}, is inserted back into the evolution equation of \autoref{eqn: BDMPS_solution_laplace_step1}, 

\begin{align}\label{eqn: BDMPS_solution_laplace_step2}
    \partial_t \tilde{F}(y,\tau) &= \tilde{F}(\nu, \tau) \, \int_0^\infty dz\, \frac{e^{-\nu z}}{z^{(3/2)}} - \tilde{F}(y,\tau) \, \int_0^\infty \frac{dz}{z^{3/2}} \nonumber \\
    &= \tilde{F}(\nu, \tau) \, \int_0^\infty dz\, \frac{(e^{-\nu z}-1)}{z^{(3/2)}} \nonumber \\
    &= \tilde{F}(\nu, \tau) \, (-2 \sqrt{\pi \nu})
\end{align}

This is a simple differential equation. From energy conservation the initial condition is \(\tilde{F}_0 = 1\) - more precisely is the initial condition a delta function which takes into account partons ending with precisely zero momentum - but the solution is for all of our purposes, 

\begin{equation}\label{eqn: BDMPS_solution_laplace_result}
    \tilde{F}(y,\tau) = e^{-2\sqrt{\pi \nu}\tau}
\end{equation}

The final step of this calculation - step 5 - is to do the inverse Laplace transformation on \autoref{eqn: BDMPS_solution_laplace_result},

\begin{align}
    F(y,\tau) &= \int_{c-i\infty}^{c+i\infty} \frac{d\nu}{2\pi i} e^{\nu y} \,\tilde{F}(\nu,\tau) \nonumber \\
    &= \frac{\tau}{y^{3/2}} \, \exp\left(-\pi \frac{\tau^2}{y}\right)
\end{align}

Reverting back \(F(y,\tau) = \sqrt{x}\, D(x,\tau)\) and \(y = 1-x\), our result is, 

\begin{equation}\label{eqn: BDMPS_solution}
    D(x,\tau ) = \frac{\tau}{\sqrt{x}(1-x)^{3/2}}\, \exp\left(-\pi \frac{\tau^2}{1-x}\right)
\end{equation}

At this point it is worth taking a deep breath, and think about what just happened. We started out with our familiar medium evolution equation, and wrote it in terms of a new variable \(\xi\), this made it possible to solve the integral in the loss term. Another change of variables allowed us to perform a Laplace transform so that the gain and loss term got he same form, and the equation could therefore be solved as an differential equation. Finally the inverse Laplace transform gave us the final expression in \autoref{eqn: BDMPS_solution}.


\subsection{Medium Splitting Functions}
For the Medium cascade the gluon splitting function is given from \cite{Energy_flow_medium_cascade_2016} as, 

\begin{equation}\label{eqn: ggg_medium_splitting_function}
    P(z) = N_c \cdot \, \frac{1}{z^{3/2}(1-z)^{3/2}}
\end{equation}

\textcolor{red}{Write about the medium splitting functions, atleast give them here}


\end{document}