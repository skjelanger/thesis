\documentclass[main.tex]{subfiles}

\begin{document}
\section{Quantum Chromodynamics}\label{sec: QCD}
The field theory of strong interactions is called quantum chromodynamics, often abbreviated \textit{QCD}. It is a non-abelian gauge theory 

The theory of strong interactions is described by the QCD Lagrangian,
\begin{align}\label{eqn: QCD_lagrangian_full}
    \mathcal{L}_{QCD} &= \mathcal{L}_{\text{quark}} + \mathcal{L}_{\text{Gluon}} + \mathcal{L}_{\text{ghost}}
\end{align}
the most striking feature of QCD is the gluon self-interaction, which is apparent when expanding the Gluonic part of the Lagrangian, 

\begin{align}
    \mathcal{L}_{\text{Gluon}} 
    &= - \frac{1}{4} F_{i\mu\nu}(x) F_i^{\mu\nu}(x) - \frac{1}{2} \left( \partial_\mu A_i^\mu(x)\right)^2 \label{eqn: QCD_Lpart_gluon}  \\
    &+ g_s f_{ijk} A_{i\mu}(x)A_{j\nu}(x) \partial^\mu A_k^\nu(x) - \frac{1}{4} g_s^2 f_{ijk} f_{ilm} A_j^\mu(x) A_k^\nu(x) A_{l\mu}(x) A_{m\nu}(x) \nonumber
\end{align}

the Feynman diagrams of the self interaction is given by \autoref{fig: feynman_three-gluon_vertex} and \autoref{fig: feynman_four-gluon_vertex}.

\begin{figure}[H]
    \centering
    \begin{minipage}{.5\textwidth}
    \centering
      \begin{tikzpicture}
            \begin{feynman}
            \vertex (a);
            \vertex [below=of a] (v);
            \vertex [below right=of v] (b);
            \vertex [below left=of v] (c);
            \diagram* {
            (a) -- [gluon, edge label = \(k_1 \) ] (v),
            (b) -- [gluon, edge label' = \(k_2 \) ] (v),
            (c) -- [gluon, edge label = \(k_3 \) ] (v),
            };
            \end{feynman}
        \end{tikzpicture}
        \caption{The three-gluon vertex}
        \label{fig: feynman_three-gluon_vertex}
\end{minipage}%
\begin{minipage}{.5\textwidth}
    \centering
        \begin{tikzpicture}
            \begin{feynman}
            \vertex (a);
            \vertex [below right=of a] (v);
            \vertex [above right=of v] (b);
            \vertex [below right=of v] (c);
            \vertex [below left=of v] (d);
            \diagram* {
            (a) -- [gluon, edge label' = \(k_1 \) ] (v),
            (b) -- [gluon, edge label = \(k_4 \) ] (v),
            (c) -- [gluon, edge label' = \(k_3 \) ] (v),
            (d) -- [gluon, edge label = \(k_2 \) ] (v),
            };
            \end{feynman}
        \end{tikzpicture}
        \caption{The four-gluon vertex}
        \label{fig: feynman_four-gluon_vertex}
\end{minipage}
\end{figure}


\begin{figure}[H]
    \centering
    \begin{minipage}{.5\textwidth}
    \centering
        \begin{tikzpicture}
            \begin{feynman}
            \vertex (a);
            \vertex [right=of a] (v);
            \vertex [above right=of v] (b);
            \vertex [below right=of v] (c);
            \diagram* {
            (a) -- [gluon, edge label = \(k_1 \) ] (v),
            (v) -- [fermion, edge label' = \(k_2 \) ] (b),
            (v) -- [anti fermion, edge label = \(k_3 \) ] (c),
            };
            \end{feynman}
        \end{tikzpicture}
        \caption{The qg -Vertex}
        \label{fig: feynman_gqq-vertex}
  \label{fig:test1}
\end{minipage}%
\begin{minipage}{.5\textwidth}
    \centering
        \begin{tikzpicture}
            \begin{feynman}
            \vertex (a);
            \vertex [right=of a] (v);
            \vertex [above right=of v] (b);
            \vertex [below right=of v] (c);
            \diagram* {
            (a) -- [fermion, edge label = \(k_1 \) ] (v),
            (v) -- [fermion, edge label' = \(k_2 \) ] (b),
            (v) -- [gluon, edge label = \(k_3 \) ] (c),
            };
            \end{feynman}
        \end{tikzpicture}
        \caption{The qq vertex}
        \label{eqn: feynman_qqg-vertex}
\end{minipage}
\end{figure}



The Vertex factors (eq 5.5 and 5.6) can be calculated from the Feynman rules, and the polarizations (5.8) from trigonometry. Putting everything together with the \(1/t\) factor from the propagator and colour factor, the matrix element squared for the \(n+1\) partons in the small angle ,can be written as, 

\begin{equation}
    |\mathcal{M}_{n+1}|^2 \sim \frac{4g^2}{t} C_A F(z;\epsilon_a,\epsilon_b,\epsilon_c) \, |\mathcal{M}_n|^2
\end{equation}



\subsection{General branching quantities}
Considering the branching of a parton \(a\) to \(b+c\) under the assumption that \(p_b^2, p_c^2 << p_a^2 \equiv t\), as illustrated in \autoref{fig: test2}.

\begin{figure}[H]
    \centering
    \feynmandiagram [horizontal=a to b] {
    a [blob] -- [plain, edge label = a] b,
    f1 [particle = b] -- [plain, edge label = \(\theta_b\)] b  -- [plain, edge label= \(\theta_c\)] f2 [particle = c],
    f1 -- [opacity=0] f2,
    };
    \caption{Test}
    \label{fig: test2}
\end{figure}

By defining the energy fraction of the branched partons as, 

\begin{equation}\label{eqn: branched_energy_fractions_ellis_5.2}
    z = \frac{E_b}{E_a} = 1- \frac{E_c}{E_a}
\end{equation}

then it can be shown in the small angle approximation \(cos \theta \approx 1- \frac{\theta^2}{2}\),

\begin{align*}
    t&= p_a^2 = p_b^2 + 2 p_b p_c + p_c^2 \\
    &= 2 (E_b E_c - |\vec p_b||\vec p_c| cos \theta )\\
    &= 2 E_b E_c (1-cos \theta) \\ 
    &= 2 \frac{E_b}{E_a} \frac{E_c}{E_a} (1-cos \theta) E_a^2 \\
    &= 2 z (1-z)  (\frac{\theta^2}{2})) E_a^2 \\
    &= z (1-z) E_a^2 \theta^2
\end{align*}

and from conservation of transverse momentum, 
\begin{equation}\label{eqn: theta_momentum_conv_ellis_5.4}
    \theta = \frac{1}{E_a} \sqrt{\frac{t}{z(1-z)}} = \frac{\theta_b}{1-z} = \frac{\theta_c}{z}
\end{equation}

\subsection{Parton branching cross sections}
To determine the cross sections for the various parton branchings, we need to consider the initial-state flux factor \(\mathcal{F}\), and final state phase space \(d\Phi_n\),

\begin{equation*}
    d\sigma_n = \mathcal{F} |\mathcal{M}_n|^2 \, d\Phi_n
\end{equation*}

Now, we are considering the scenario where parton a is branching into parton b and c so we must make the following replacement, 

\[d\Phi_n = \cdots \, \frac{d^3 \vec p_a}{2(2\pi)^3 E_a} \quad \Rightarrow \quad d\Phi_{n+1} = \cdots \, \frac{d^3 \vec p_b}{2 (2\pi)^3 E_b} \, \frac{d^3 \vec p_c}{2(2\pi)^3 E_c}\]

 with the criteria that \(p_c = p_a - p_b\), we can find in the small-angle approximation that, 
 
 \begin{align*}
     d\Phi_{n+1} &= \frac{d^3 \vec p_b}{(2\pi)^3 2E_b} \frac{d^3 \vec p_c}{(2\pi)^3 2E_c} \\
     &=\frac{d^3 \vec p_b}{(2\pi)^3 2E_b} \frac{d^3 \vec p_a}{(2\pi)^3 2E_a} \frac{1}{1-z} \\
     &= d\Phi_n \frac{d^3 \vec p_b}{(2\pi)^3 2E_b} \frac{1}{1-z} \\
     &= d\Phi_n \frac{1}{2(2\pi)^3} \int\frac{E_b^2 (dE_b\, \theta_b d\theta_b \, d\phi )}{E_b} \frac{1}{1-z} \\
     &= d\Phi_n \frac{1}{2(2\pi)^3} \int E_b dE_b\, \theta_b d\theta_b \, d\phi \frac{1}{1-z}
 \end{align*}
 
 inserting, \(dt \, \delta( t-2E_bE_c(1-\theta) = dt \, \delta( t- E_bE_c \theta^2)\), and \(dz \delta(z- \frac{E_b}{E_a})\), into the integral, the result should remain the same
 
\begin{align*}
    d\Phi_{n+1} &= d\Phi_n \frac{1}{2(2\pi)^3} \int E_b dE_b\, \theta_b d\theta_b \, d\phi \frac{1}{1-z}  dt \, \delta( t- E_bE_c \theta^2) dz \delta(z- \frac{E_b}{E_a}) \\
    &= \cdots \\
    &= \cdots \\
    &= d\Phi_n \frac{1}{4 (2\pi)^3} \, dt\, dz \, d\phi 
\end{align*}

and we can calculate the cross section for the splitting, starting from equation 5.22 in the book,

\begin{equation*}
    d\sigma_n = \mathcal{F} |\mathcal{M}_n|^2 d\Phi_n
\end{equation*}

thus, 

\begin{align*}
    d\sigma_{n+1} &= \mathcal{F} |\mathcal{M}_{n+1}|^2 d\Phi_{n+1}\\
    &= \mathcal{F} \left( \frac{2g^2}{t} C_F|\mathcal{M}_n|^2 \right) \left( d\Phi_n \frac{1}{4 (2\pi)^3} \, dt\, dz \, d\phi  \right) \\
    &= d\sigma_n  \left( \frac{g^2}{t} C_F\right) \left( \frac{1}{2 (2\pi)^3} \, dt\, dz \, d\phi  \right) \\
    &= d\sigma_n \, \frac{dt}{t}\, dz \, \frac{d\phi}{2\pi} \left( \frac{g^2}{2(2\pi)^2} \right) C_F \\
    &= d\sigma_n \, \frac{dt}{t}\, dz \, \frac{d\phi}{2\pi} \left( \frac{(\alpha_S 4\pi)}{2(2\pi)^2} \right) C_F \\
    &= d\sigma_n \, \frac{dt}{t}\, dz \, \frac{d\phi}{2\pi} (\frac{\alpha_S }{2\pi} ) C_F
\end{align*}



replacing, \(\hat{P}_{ba}(z) = \int \frac{d\phi}{2\pi}\), we get, 

\begin{equation}
    d\sigma_{n+1}= d\sigma_n \, \frac{dt}{t}\, dz \, (\frac{\alpha_S }{2\pi} ) \hat{P}_{ba}(z)
\end{equation}

Similar for spacelike branchings (page 165).


\end{document}