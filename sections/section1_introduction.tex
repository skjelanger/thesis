\documentclass[main.tex]{subfiles}

\begin{document}
The project deals with calculating the energy loss of leading jets or partons that traverse a hot and dense quark-gluon plasma. While inclusive distributions account for multiple branchings along the path through a medium, for this application we will follow in each step the branch that retains the largest energy fraction.

For jets in electron-positron or proton-proton collisions, referred loosely to as "vacuum", the evolution from a very large jet off-shellness down to hadronization, that leads to multiple emissions, is described by the famous DGLAP evolution equations. We will introduce these evolutin equations, and solve them analytically and numerical (using Monte Carlo techniques).

The goal of the project is to calculate the energy lost by a leading particle in a medium-induced cascade. This will be formulated analytically, and attempted to solve it using standard techniques (Laplace transform). In parallel, the candidate will build a Monte Carlo routine to generate branchings and to, ultimately, compute the desired observables.

\end{document}