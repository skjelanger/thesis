\documentclass[main.tex]{subfiles}

\begin{document}

\chapter*{Introduction}\label{cpt:intro}
\addcontentsline{toc}{part}{\nameref{cpt:intro}}
Shortly after The Big Bang, the universe consisted exclusively of fundamental particles in a primordial soup called quark-gluon plasma (QGP). Quarks and gluons would exist in this matter as free particles, not being bound by the strong nuclear force to form protons and neutrons.
Colliders such as the Relativistic Heavy Ion Collider (RHIC) based at the Brookhaven National Laboratory and the Large Hadron Collider (LHC) at CERN, have allowed us to create quark-gluon plasma in the modern world. Heavy ions such as \(Au\) and \(Pb\) are accelerated to velocities approaching the speed of light and smashed together in relativistic heavy-ion collisions. In the aftermath of these collisions, the temperature and density is large enough for QGP to form, allowing us to model the first moments of the universe.

Measuring the properties of the quark-gluon plasma in colliders is exceedingly difficult. Not only does the QGP exist for just a brief moment (\(10^{-24}s\)), but how would one go ahead and measure something existing at the center of relativistic heavy-ion collisions? The answer is \emph{jets}. Jets are collimated groups of hadrons generated by successive branchings of a highly energetic parton (quark or gluon), created in relativistic heavy-ion collisions. When this energetic parton traverses the quark-gluon plasma it interacts with the medium, affecting the distribution of final hadrons we observe in the detectors. Therefore, jets provide a way of probing the medium as we know the initial conditions, and can determine the medium impact by measuring the parton showers in the detector. An analogy can be made by considering a game of chess. Imagine you walk into an empty room with a chessboard in the middle of a game. Assuming we know how the pieces were set up at the start of the game, it should be possible to determine which moves were made to get the pieces to where they are now. The challenge is doing this thousands of times, while not being entirely sure what the rules of the game actually are.

This thesis is a study of how energetic partons, created in the aftermath of heavy-ion collisions, split or branch into pairs of new partons, with and without a background medium. Two different fragmentation scenarios will be considered. The first one is the inclusive parton distribution where we are interested in all of the partons accumulated throughout the branching process, such that the total energy is conserved. In the second scenario, we will focus on the leading parton, in which we follow the parton with the highest energy in each branching, and the other partons are considered a loss of energy from the leading parton.

Both an analytical perspective and a numerical perspective will be presented in this thesis. The former relies on known literature and published papers to formulate the current understanding of parton branching for vacuum and medium cascades. The evolution equations for both cascades will be presented, and the most important results will be recreated. The numerical perspective of the thesis is concentrated on developing Monte-Carlo programs which rely on randomly generated numbers, to create parton showers by iterating through the evolution equations. The distributions generated from these programs will be compared to the analytical results, and the properties of the two cascades will be discussed and highlighted using plots.

The thesis is structured in different chapters, with their own purpose. \autoref{cpt:fun} is the \textit{Fundamentals}, which will present the fundamental theory and concepts which should be known for truly understanding the content of the thesis. Its primary function is to revive old knowledge, and formal derivations will generally not be given. The first topic to be covered is the foundations of quantum chromodynamics (QCD), which is the theory of the strong nuclear force. Following that, an introduction to jet evolution and jet observables will be given. Finally, we will introduce the most basic quantities of parton branching, which will be used throughout the thesis.

\autoref{cpt:ana} is \textit{Analytical}. Here all of the mathematics and formalisms of parton branchings will be discussed, focusing on the inclusive parton distribution. Beginning with parton branching in vacuum, where the DGLAP evolution equations will be presented, alongside the Sudakov form factor, and analytical solution. The same structure is given for parton branching in medium, where the evolution equations are the in-medium kinetic rate equations. The corresponding in-medium Sudakov form factors will also be introduced, and a solution to the evolution equations will be calculated.

\autoref{cpt:num} is \textit{Numerical}. Here the analytical concepts will be used to creating a Monte-Carlo program for simulating parton showers. For vacuum we will be simulating cascades consisting exclusively of gluons, and cascades consisting of both quarks and gluons. When simulating medium cascades, we will be working exclusively with gluons. The results of these programs are then plotted alongside the analytical solutions, and used to discuss and highlight the properties of the different cascades. \autoref{cpt:ana} and \autoref{cpt:num} will therefore complement each other. 

\autoref{cpt:leading}, \textit{Leading Parton and Energy-Loss}, is dedicated to the leading parton formalism. While the inclusive parton distribution is well known, there is still a lot to learn about the leading parton distribution. The key concepts and obstacles will be presented using current models for the energy-loss. We will then determine how often the leading parton remains on-branch, and will also formulate a new set of evolution equations for the leading parton and attempt to solve them.

The results we are looking for in \autoref{cpt:leading} might be an important step in understanding the leading parton distribution and its energy-loss. A deeper understanding of the leading parton distribution could refine current studies of jet quenching in heavy-ion collisions, and develop new observables sensitive to the properties of QCD.

\end{document}